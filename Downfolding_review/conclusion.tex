\section{Conclusion and Future prospects}
We have explained the DMD method that we have been developing, primarily in conjunction with the 
\textit{ab-initio} QMC approach. The practical motivation is to take data from first principles (continuum) 
calculations as inputs for lattice model (discrete) methods. 
Since one is mapping a many-body problem to a few-body one, the relevant quantities of interest are 
the reduced density matrices associated with the many-body wavefunctions projected to a one-body space. 
The density matrix based approach is appealing as it is democratic 
in the determination of the hopping and interaction parts of the effective Hamiltonian, and provides multiple 
checks on its validity. We have discussed representative examples to present the conceptual and algorithmic aspects of DMD. 

The resultant lattice model can be efficiently and accurately solved for large system sizes~\cite{LeBlanc_PRX} 
using techniques designed and suited for small local Hilbert spaces; these include exact or 
selected diagonalization~\cite{DeRaedt,Tubman_selci,Holmes_Tubman_Umrigar}, 
density matrix renormalization group (DMRG)~\cite{White1992}, tensor networks~\cite{PEPS,Changlani_CPS,NeuscammanCPS}, 
dynamical mean field theory (DMFT)~\cite{Kotliar2006}, density matrix embedding (DMET)~\cite{DMET_2012} and 
lattice QMC methods~\cite{Scalapino, Trivedi_Ceperley, Zhang_AFQMC, Sandvik_loops, Prokofiev, 
Booth2009,SQMC,Holmes_Changlani_Umrigar, Booth2013}. These methods have also been used to obtain excited states, 
dynamical correlation functions and thermal properties, that have been difficult to obtain in \textit{ab-initio} approaches. 

What sort of accuracy should one expect with effective Hamiltonians and DMD? The holy grail of quantum chemistry 
and electronic structure is to obtain an energy of 1 mHa (0.027 eV) per atom; this 
remains an open challenge despite decades of work. The effective Hamiltonian approach 
ameliorates this problem, only relative energies are important, and 
the total energy is not of particularly fundamental interest. Our experience with DMD in the present and past work~\cite{Changlani2015} 
suggests excitation spectra can be accurately determined and potentially improved with more \textit{ab-initio} data 
and more refined effective Hamiltonians. However, it is clear that the entire approach is contingent on the 
cancellation of the large energy associated with the core electrons, 
whose role is to primarily renormalize the interactions between the active (valence) electrons. 
This aspect can be a limitation when the core and active electrons are strongly entangled i.e. 
when the separation of energy scales is not prominent, in which case an energy-dependent description is needed. 
Thus, DMD and downfolding in general is possibly best suited for extended systems (solids) where this separation exists. 

Finally we would like to emphasize that DMD, though conceptually simple, 
is still a method in its development stages, with room for algorithmic improvements and new applications. 
Advances in the field of inverse problems~\cite{Berg2017} could be incorporated into DMD to 
mitigate the problems associated with optimization and over-fitting. 
Here we briefly outline some aspects that need further research,
\begin{itemize} 
	\item Construction of wavefunction database:
	The DMD method relies crucially on the availability of a low energy space of \textit{ab initio} wavefunctions, 
	which despite not being eigenstates, reveal the nature of the effective Hamiltonian. Automating its construction 
	remains challenging. We propose Slater Jastrow wavefunctions and deforming the orbitals entering the Slater 
	determinant as one way of probing the low energy manifold.
	\item Optimal choice of active orbitals (one body space):
	In the present work, we chose DFT orbitals (in a certain energy window) and localized them to get the one body orbitals. 
	The choice could only be justified based on inspecting the trace of the 1-RDM and ensuring that it 
	equalled the expected number of electrons. We expect to use intrinsic atomic orbitals (pointed out to us by G. K. Chan) which 
	mitigate this issue. 
	\item Form of the low energy model Hamiltonian:
	Extensive work in the literature has been devoted to parameterizing compact effective Hamiltonians~\cite{Georges, Oles, Coury}. 
	Since the low energy effective Hamiltonian is not unique, there is not necessarily one right way of downfolding. 
	One could catalogue all functional forms based on previous parameterizations 
	or there may be some merit in determining the minimal description with fewest non zero parameters.
	Most models for strong correlation \textit{assume} two body forms which appears to work well in practice, 
	but is not guaranteed on mathematical grounds. (Even though the Coulomb operator is two body, 
	its low energy description need not be.) In this regard, DMD could also be used to assess 
	the accuracy of commonly used Hamiltonians and associated parameter sets.
	\item Other (non QMC) wavefunction based electronic structure methods:
	There are several wavefunction based quantum chemistry 
	methods for electronic structure like coupled cluster and ab-initio DMRG which work primarily 
	directly in orbital space. These could also be potentially used in conjunction with DMD.
\end{itemize} 

In terms of applications, we hope the method will be eventually applied to solids involving transition 
metals and/or rare earth elements, which are in the strongly correlated regime due to the presence of localized orbitals. 
Some challenging areas include,
\begin{itemize} 
	\item Spin models for magnetism: 
	The magnetism associated with strongly correlated Mott insulators, compounded by effects such as geometrical frustration, 
	is highly non trivial and has seen immense research activity in the last few years. 
	For example, herbertsmithite harbors an exotic topological phase of matter - a "quantum spin liquid"~\cite{NormanRMP}; 
	lattice simulations of idealized kagome Heisenberg models~\cite{Yan_Huse_White, Changlani_kagome} have built evidence 
	for such a state amongst a zoo of many other low energy candidates. Thus, 
	downfolding such materials~\cite{Jeschke2013} appears to be a useful way for accessing low energy scales. 
	We caution that energy scales associated with magnetism can be in the meV range, and no electronic structure 
	method currently offers the accuracy needed to directly resolve such small energy differences without unbiased approximations. 
	It may thus be necessary to perform multiple downfolding steps to map from the Schroedinger equation to a spin model.
	\item Multi-band models for superconductors:
	As previously mentioned, unconventional (non BCS) superconductivity in materials such as the cuprates and iron based 
	pnictides has fuelled the study of strongly correlated systems.
	Several parameter sets exist in the literature for multi-band models, but it appears that there is little universal consensus. 
	We are optimistic that additional \textit{ab-initio} inputs from DMD will help constrain the part of parameter 
	space relevant for these materials. Trends of the parameters, and the effectiveness of the multi-band model itself, 
	with pressure dependence and doping remain largely unexplored.  
\end{itemize} 
In summary, the DMD method offers a new route to connecting the worlds of lattice models and \textit{ab-initio} approaches. 
We believe making this relationship precise will be crucial for a detailed and definitive 
understanding of strongly correlated quantum materials. 

\section{Acknowledgements} 
We thank  David Ceperley,  Richard Martin, Cyrus Umrigar,  Garnet Chan,  Shiwei Zhang, Steven White,  
Lubos Mitas, So Hirata, Bryan Clark, Norm Tubman, Miles Stoudenmire and Victor Chua for extremely useful and insightful discussions. 
This work was funded by the grant DOE FG02-12ER46875 (SciDAC). HJC acknowledges support from the U.S. Department of Energy, 
Office of Basic Energy Sciences, Division of Materials Sciences and Engineering under Award DE-FG02-08ER46544 for his work at the Institute for Quantum Matter (IQM). 

\section*{Author Contributions}
HJC, HZ and LKW conceived the initial DMD ideas and designed the project and organization of the paper. 
All authors contributed to the theoretical developments and various representative \textit{ab-initio} and lattice examples. 
All authors contributed to the analysis of the data, discussions and writing of the manuscript. 
LKW oversaw the project. HJC and HZ contributed equally to this work.
 
\section*{Additional Information}
Competing financial interests: The authors declare no competing financial interests.
