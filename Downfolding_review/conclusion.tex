\section{Conclusion and Future prospects}



The density matrix downfolding (DMD) technique uses data derived from low-energy approximate solutions to a high energy Hamiltonian to systematically determine an effective Hamiltonian that describes the low-energy behavior of the system.
It is based on several rather simple proofs which occupy a role similar to the variational principle; they allow us to know which effective models are closer to the correct solution than others. 
The method is very general and does not require a quasiparticle picture to apply, and neither does it have double-counting issues.
It treats all interactions on an equal footing, so hopping parameters are naturally modified by interaction parameters and so on.
While most of the applications have used the first principles quantum Monte Carlo method to obtain the low-energy solutions, the method is completely general and can be used with any solution method that can produce high quality energy and reduced density matrices.  
We have discussed several examples to present the conceptual and algorithmic aspects of DMD. 

The resultant lattice model can be efficiently and accurately solved for large system sizes~\cite{LeBlanc_PRX} using techniques designed and suited for small local Hilbert spaces; these include exact or selected diagonalization~\cite{DeRaedt,Tubman_selci,Holmes_Tubman_Umrigar}, density matrix renormalization group (DMRG)~\cite{White1992}, tensor networks~\cite{PEPS,Changlani_CPS,NeuscammanCPS}, dynamical mean field theory (DMFT)~\cite{Kotliar2006}, density matrix embedding (DMET)~\cite{DMET_2012} and lattice QMC methods~\cite{Scalapino, Trivedi_Ceperley, Zhang_AFQMC, Sandvik_loops, Prokofiev, 
Booth2009,SQMC,Holmes_Changlani_Umrigar, Booth2013}. 
These methods have also been used to obtain excited states, dynamical correlation functions and thermal properties, that have been difficult to obtain in \textit{ab initio} approaches. 


DMD, though conceptually simple, is still a method in its development stages, with room for algorithmic improvements and new applications. 
Advances in the field of inverse problems~\cite{Berg2017} could be incorporated into DMD to 
mitigate the problems associated with optimization and over-fitting. 
Here we briefly outline some aspects that need further research:
\begin{enumerate}
	\item The wave function database ($\ket{\Psi} \in {\mathcal LE}$):
	The DMD method relies crucially on the availability of a low energy space of \textit{ab initio} wave functions. While these wave functions do not have to be eigenstates, automating their construction remains challenging and realistically requires knowledge of the physics to be described.
	\item Optimal choice of basis functions. The second-quantized operators in the effective Hamiltonian correspond to a basis in the continuum. The quality of the model depends on the basis describing the changes between low-energy wave functions accurately.
	\item Form of the low energy model Hamiltonian. While the exact effective Hamiltonian is unique, there may be many ways of approximating it with varying levels of compactness and accuracy.
\end{enumerate} 
The advantage of the DMD framework is that all these can be resolved internally.
Given a good sampling of ${\mathcal LE}$, (2) and (3) can be resolved using regression. 
Given that (2) and (3) are correct or near correct, then (1) can be resolved by finding binding planes, as noted in Section~\ref{sec:theory}.
The method thus has a degree of self consistency; it will return low errors only when 1-3 are correct.

We have shown applications to strongly correlated models (3-band), {\it ab initio} bulk systems hydrogen chain and graphene, and a transition metal molecule FeSe.
The technique is on the verge of being applied to transition metal bulk systems; there are no major barriers to this other than a polynomially scaling computational cost and the substantial amount of work involved in parameterizing and fitting models to these systems.
Looking into the future, we anticipate that this technique can help with the definition of a correlated materials genome--what effective Hamiltonian best describes a given material is highly relevant to its behavior.


%In terms of applications, we hope the method will be eventually applied to solids involving transition 
%metals and/or rare earth elements, which are in the strongly correlated regime due to the presence of localized orbitals. 
%Some challenging areas include,
%\begin{itemize} 
%	\item Spin models for magnetism: 
%	The magnetism associated with strongly correlated Mott insulators, compounded by effects such as geometrical frustration, 
%	is highly non trivial and has seen immense research activity in the last few years. 
%	For example, herbertsmithite harbors an exotic topological phase of matter - a "quantum spin %liquid"~\cite{NormanRMP}; 
%	lattice simulations of idealized kagome Heisenberg models~\cite{Yan_Huse_White, Changlani_kagome} have built evidence 
%	for such a state amongst a zoo of many other low energy candidates. Thus, 
%	downfolding such materials~\cite{Jeschke2013} appears to be a useful way for accessing low energy scales. 
%	We caution that energy scales associated with magnetism can be in the meV range, and no electronic structure 
%	method currently offers the accuracy needed to directly resolve such small energy differences without unbiased approximations. 
%	It may thus be necessary to perform multiple downfolding steps to map from the Schroedinger equation to a spin model.
%	\item Multi-band models for superconductors:
%	As previously mentioned, unconventional (non BCS) superconductivity in materials such as the cuprates and iron based 
%	pnictides has fuelled the study of strongly correlated systems.
%	Several parameter sets exist in the literature for multi-band models, but it appears that there is little universal consensus. 
%	We are optimistic that additional \textit{ab initio} inputs from DMD will help constrain the part of parameter 
%	space relevant for these materials. Trends of the parameters, and the effectiveness of the multi-band model itself, 
%	with pressure dependence and doping remain largely unexplored.  
%\end{itemize} 
%In summary, the DMD method offers a new route to connecting the worlds of lattice models and \textit{ab initio} approaches. 
%We believe making this relationship precise will be crucial for a detailed and definitive understanding of strongly correlated quantum materials. 

\section{Acknowledgements} 
We thank  David Ceperley,  Richard Martin, Cyrus Umrigar,  Garnet Chan,  Shiwei Zhang, Steven White,  
Lubos Mitas, So Hirata, Bryan Clark, Norm Tubman, Miles Stoudenmire and Victor Chua for extremely useful and insightful discussions. 
This work was funded by the grant DOE FG02-12ER46875 (SciDAC). HZ acknowledges support from Argonne Leadership Computing Facility, a Department of Energy Office of Science User Facility under Contract DE-AC02- 06CH11357. HJC acknowledges support from the U.S. Department of Energy, 
Office of Basic Energy Sciences, Division of Materials Sciences and Engineering under Award DE-FG02-08ER46544 for his work at the Institute for Quantum Matter (IQM). 
This research is part of the Blue Waters sustained-petascale computing project, which is supported by the National Science Foundation (awards OCI-0725070 and ACI-1238993) and the state of Illinois. Blue Waters is a joint effort of the University of Illinois at Urbana-Champaign and its National Center for Supercomputing Applications.

\section*{Author Contributions}
HJC, HZ and LKW conceived the initial DMD ideas and designed the project and organization of the paper. 
All authors contributed to the theoretical developments and various representative \textit{ab initio} and lattice examples. 
All authors contributed to the analysis of the data, discussions and writing of the manuscript. 
LKW oversaw the project. HJC and HZ contributed equally to this work.
 
\section*{Additional Information}
Competing financial interests: The authors declare no competing financial interests.
