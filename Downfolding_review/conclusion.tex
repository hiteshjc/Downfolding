\section{Conclusion and Future prospects}
To summarize, we have explained the AIDMD method that we have been developing, primarily in conjunction with the 
\emph{ab-initio} QMC approach. The practical motivation is to take data from first principles (continuum) 
calculations as inputs for lattice model (discrete) methods. 
Since one is mapping a many body problem to a few body one, the relevant quantities of interest are 
the reduced density matrices associated with the many-body wavefunctions projected to a one-body space. 
The density matrix based approach is appealing as it is democratic 
in the determination of the hopping and interaction parts of the effective Hamiltonian, and provides multiple 
checks on its validity. We have discussed representative examples to present the conceptual and algorithmic aspects of AIDMD. 

What sort of accuracy should one expect with effective Hamiltonians and AIDMD? The holy grail of quantum chemistry 
and electronic structure is to obtain an energy of 1 mHa (0.027 eV) per atom; this 
remains an open challenge despite decades of work. The effective Hamiltonian approach somewhat ameliorates 
this problem, only relative energies are important, neither the total energy (nor its accuracy) is of particularly 
fundamental interest. This is contingent on the cancellation of the large energy associated with the core electrons, 
whose role is to primarily renormalize the interactions between the active (valence) electrons. This aspect can also 
be a major limitation when the core and active electrons are strongly entangled i.e when the separation of energy 
scales is not prominent (in which case an energy-dependent description may be needed). 
Thus, it is possibly best suited for extended systems (solids) where this separation exists. 
In our view, AIDMD (and downfolding in general) opens up many avenues for determination of physical 
quantities not easily calculated in ground state approaches. Our experience suggests excitation spectra 
can be accurately determined to 0.2 eV (or less)~\cite{Changlani2015} and potentially improved with 
more \emph{ab-initio} data and more refined effective Hamiltonians.  

Finally we would like to emphasize that AIDMD, though conceptually simple, 
is still a method in its development stages, with room for algorithmic improvements and new applications. It suffers 
from the usual problems of optimization and overfitting, but with advances in these fields we believe some of these major 
concerns may become less problematic. Some of the issues that need further research are,
\begin{itemize} 
	\item Construction of wavefunction database:
	The AIDMD method relies crucially on the availability of a low energy space of \emph{ab initio} wavefunctions, 
	which despite not being eigenstates, reveal the nature of the effective Hamiltonian. Automating its construction 
	remains somewhat challenging. We propose Slater Jastrow wavefunctions and deforming the orbitals entering the Slater 
	determinant as one way of probing the low energy manifold.
	\item Optimal choice of active orbitals (one body space):
	In the present work, we chose DFT orbitals (in a certain energy window) and localized them to get the one body orbitals. 
	The choice could only be justified based on looking at the trace of the 1-RDM and ensuring that it 
	equalled the expected number of electrons. We expect to use intrinisic atomic orbitals (pointed out to us by G. K. Chan) which 
	mitigate this issue. 
	\item Form of the low energy model Hamiltonian:
	Extensive work in the literature has been devoted to parameterizing compact effective Hamiltonians~\cite{Georges, Oles, Coury}. 
	Since the low energy effective Hamiltonian is not unique, there is not necessarily one right way of downfolding. 
	One could catalogue all functional forms based on previous parameterizations 
	or there may be some merit in determining the minimal description with fewest non zero parameters.
	Note that most models for strong correlation \emph{assume} two body forms. While this appears to work well in practice, 
	it is by not guaranteed on mathematical grounds. (Even though the Coulomb operator is two body, 
	its low energy description need not be.)
	\item Other (non QMC) wavefunction based electronic structure methods:
	There are several wavefunction based quantum chemistry 
	methods for electronic structure like coupled cluster and ab-initio DMRG which work primarily 
	directly in orbital space. These could also be potentially explored in conjunction with AIDMD.
\end{itemize} 

In terms of applications, we hope the method will be eventually applied to solids involving transition 
metals and/or rare earth elements, which are in the strongly correlated regime due to the presence of localized orbitals. 
Some challenging areas include,
\begin{itemize} 
	\item Spin models for magnetism: 
	The magnetism associated with strongly correlated Mott insulators, compounded by effects such as geometrical frustration, 
	is highly non trivial and has become an active area of research in its own right. 
	%This is compounded by 
	%the presence of geometric frustration, impurities and defects and 
	%spin-orbital effects. 
	For example, the two dimensional kagome lattice planes in 
	herbertsmithite harbor an exotic topological phase of matter - a "quantum spin liquid". While our understanding is primarily 
	from Heisenberg models~\cite{Yan_Huse_White, Changlani_kagome}, 
	contentious issues plague a complete understanding and connection to experiment (there has been recent progress 
	in downfolding herbertsmithite~\cite{Jeschke}). 
	Other areas of application include pyrochlore iridates and titanates and Kitaev materials. 
	We caution that relevant energy scales associated with magnetism can be $0.1$ meV or less, and no electronic structure 
	method is currently that accurate. It may thus be necessary to perform multiple downfolding steps to map from 
	the Schrodinger equation to a spin model.
	\item Multiband models for superconductors:
	As previously mentioned, unconventional (non BCS) superconductivity in materials such as the cuprates and iron based 
	pntictides has fuelled the study of strongly correlated systems.
	Several parameter sets exist in the literature for multi-band models, but it appears that there is little universal consensus. 
	We are optimistic that additional \emph{ab-initio} inputs from AIDMD will help constrain the part of parameter 
	space relevant for these materials. Trends of the parameters (and the effectiveness of the three band 
	or one band model itself) with pressure dependence and doping remain largely unexplored.  
	%\item Quantum computing (\HJC{tell me if this one is too weird}) : 
	%Quantum computers have been heralded as the future of computing with many promising advances. 
	%There has been progress in the simulation of small molecules in compact basis sets, but scalability (needed for solids) 
	%remains an issue. Downfolding macould help in the reduction in the number of active orbitals 
\end{itemize} 

\section{Acknowledgement} 
We thank  David Ceperley,  Richard Martin, Cyrus Umrigar,  Garnet Chan,  Shiwei Zhang, Steven White,  
Lubos Mitas, So Hirata, Bryan Clark, Norm Tubman, Miles Stoudenmire and Victor Chua for extremely useful and insightful discussions. 
This work was funded by the grant DOE FG02-12ER46875 (SciDAC). HJC acknowledges support from the U.S. Department of Energy, 
Office of Basic Energy Sciences, Division of Materials Sciences and Engineering under Award DE-FG02-08ER46544 for his work at the Institute for Quantum Matter (IQM). 

\section*{Author Contributions}
HJC, HZ and LKW conceived the initial AIDMD ideas and designed the project and organization of the paper. 
All authors contributed to the theoretical developments and various representative \textit{ab-initio} and lattice examples. 
All authors contributed to the analysis of the data, discussions and writing of the manuscript. 
LKW oversaw the project. HJC and HZ contributed equally to this work.
 
\section*{Additional Information}
Competing financial interests: The authors declare no competing financial interests.
