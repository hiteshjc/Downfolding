\subsection{One dimensional hydrogen chain}
We now move to one of the simplest extended \emph{ab-initio} systems, a hydrogen chain with periodic boundary conditions. 
In the absence of availability of exact eigenstates, the example will highlight the effectiveness of the N-AIDMD approach.
We consider the case of $10$ atoms and work in a regime where the inter-atomic distance $r$ is 
relatively large ($r=1.5$ to $3.0$ \AA), such that the system is potentially well described by a 1-band Hubbard model 
in terms of primarily $s$-like orbitals, whose form we discuss here. 

For a given $r$, we first obtain single-particle Kohn-Sham orbitals from a set of spin-unrestricted and 
spin-restricted DFT-PBE calculations. The localized orbital basis upon which the descriptors (density matrices) 
are calculated is obtained by generating intrinsic atomic orbitals (IAO) from the Kohn-Sham orbitals, and orthogonalizing them using 
the L\"owdin procedure; our results are shown in Fig.~xx \HHZ{Could you plot these orbitals and replace the Wannier orbitals stuff?}
These form the effective orbitals that enter the 1-band Hubbard Hamiltonian. 

To generate a database of wavefunctions needed for the N-AIDMD, 
we produce a set of wavefunctions (\HHZ{Slater-Jastrow wave functions?}) consisting of singles- and doubles- excitations 
to the Slater determinant, 
\begin{subequations}
\begin{eqnarray}
| s \rangle = & \Big[a^\dagger_{i \sigma} a_{k \sigma}   | KS \rangle \Big]e^J \\
| d \rangle = & \: \Big[a^\dagger_{i \sigma} a^\dagger_{j \sigma'} a_{k \sigma'} c_{l \sigma}   | KS \rangle\Big]e^J ,
\end{eqnarray}
\end{subequations}
where $|KS\rangle$ is the Stater determinant of occupied Kohn-Sham orbitals, $\sigma$ and $\sigma'$ are spin indices, 
and $a_{i}^\dagger$ ($a_{i}$) is a single-electron creation (destruction) corresponding to Kohn-Sham orbitals, 
and $e^J$ is the Jastrow factor which was optimized using variational Monte Carlo. 

\begin{figure}
\centering
\includegraphics[scale=0.5]{./Figures/rms_ut_error_vs_separation_h_chain.pdf}
\includegraphics[scale=0.5]{./Figures/rms_ut_error_vs_separation_h_chain.pdf}
\caption{\HJC{Put model vs ab-initio plots here}The RMS error in the fitted $U$-$t$ model for the periodic H$_{10}$ chain, relative to the \textit{ab-initio} energies. The RMS error is less than 1 eV for sufficiently long bond lengths.}\label{fig:fit_quality}
\end{figure}
 

Having computed the energies (expectation values of the Hamiltonian) and corresponding 
descriptors for these wavefunctions, and having verified the independence of the $U$ and $t$ descriptors, 
we fit the 1-band Hubbard Hamiltonian using least-squares fitting, as part of the formalism described in sec. 2. 
Fig.~\ref{fig:fit_quality} shows fits for two representative $r$. As expected, the RMS error 
is significantly smaller for larger $r$, verifying the effectiveness of the approach in this limit. 

In addition, we show trends of the fitted value of the one-body hopping $t$ and the repulsion $U$ as a function of $r$ 
in Fig.~\ref{fig:Parameters-vs-Bond-t}. Consistent with physical intuition, $t$ decreases towards zero at larger $r$
and the value of $U/t$ rises. Based on these values and the extensive body of work in 1D systems we expect a Luttinger liquid 
to insulator transition to occur at $U/t=...$ corresponding to $r=...$.  

\begin{figure}
\centering
\includegraphics[scale=0.5]{./Figures/$t$_vs_separation_h_chain_ols.pdf}
\includegraphics[scale=0.5]{./Figures/$t$_vs_separation_h_chain_ols.pdf}
\caption{\HJC{Put t vs r and U/t vs r} The one-body hopping $t$ parameter as a function of lattice constant for the periodic H$_{10}$ chain, obtained from a fitted $U$-$t$ model. The parameter value declines to zero as the lattice constant increases. \HHZ{I suggest to remove the titles from the figure, and put description in the caption instead. The y-axis label: how about "Hopping parameter $t$"}}\label{fig:Parameters-vs-Bond-t}

\end{figure}
