\subsection{One dimensional hydrogen chain}
\label{subsection:1dhydrogen}
We now move on to one of the simplest extended \emph{ab initio} systems, a hydrogen chain in one dimension with periodic boundary conditions. The one-dimensional hydrogen chain has been used as a model for validating a variety of modern \textit{ab initio} many-body methods \cite{H10_Simons}. 
We consider the case of $10$ atoms with periodic boundary conditions and work in a regime where the inter-atomic distance $r$ is in the range $1.5 - 3.0$ \AA, such that the system is  well described in terms of primarily $s$-like orbitals. 

For a given $r$, we first obtain single-particle Kohn-Sham orbitals from a set of spin-unrestricted and 
spin-restricted DFT-PBE calculations. The localized orbital basis upon which the RDMs (descriptors) 
are evaluated is obtained by generating intrinsic atomic orbitals (IAO) \cite{knizia_intrinsic_2013} from the Kohn-Sham orbitals 
orthogonalized using the L\"owdin procedure (see Figure~\ref{fig:fit_quality}). These are the orbitals that enter the one-band Hubbard Hamiltonian. 
%\BDB{Need citaton? - Yep, done,  K} 
Then, to generate a database of wavefunctions needed for the DMD, we produce a set of Slater-Jastrow 
wavefunctions consisting of singles and doubles excitations to the Slater determinant:
\begin{subequations}
\begin{eqnarray}
| s \rangle = & e^J \Big[a^\dagger_{i \eta} a_{k \eta}   | KS \rangle \Big] \,,\\
| d \rangle = & \: e^J \Big[a^\dagger_{i \eta} a^\dagger_{j \eta'} a_{k \eta'} a_{l \eta}   | KS \rangle\Big] ,
\end{eqnarray}
\end{subequations}
where $|KS\rangle$ is the Slater determinant of occupied Kohn-Sham orbitals, $\eta \neq \eta'$ are spin indices, 
%\lucas{it is unclear here if $\eta$ and $\eta'$ are allowed to be different. }
and $a_{i}^\dagger$ ($a_{i}$) is a single-electron creation (destruction) operator corresponding to a particular Kohn-Sham orbital. The $k,l$ indices label occupied orbitals in the original Slater determinant, while $i,j$ are virtual orbitals. 
$e^J$ is a Jastrow factor optimized by minimizing the variance of the local energy. 

We compute the energies (expectation values of the Hamiltonian) and the RDMs on each wave function within DMC. 
By computing the trace of the resulting 1-RDMs, we verify that all the electrons present in the system are represented within the localized basis of $s$-like orbitals. 
If the trace of the 1-RDM deviates from the nominal number of electrons for a particular state by more than some chosen threshold - 2\% in this example -
%lucas{be more specific. What are the thresholds? - Done, K} 
it indicates that some orbitals are occupied ($2s$- or $2p$-like orbitals for hydrogen)
%\lucas{could be 2s -  Sure, K}
that are not represented within the localized IAO basis used for computing the descriptors. 
Hence, these states do not exist within the $\mathcal LE$ space, and cannot be described by a one-band $s$-orbital model. We exclude such states from the wave function set. 
 %\lucas{Are these very high energy states? - Reworded to be more clear}
The acquired data is then used in DMD to downfold to a one-band Hubbard Hamiltonian.
\renewcommand{\subfigimg}[3][,]{%
  \setbox1=\hbox{\includegraphics[#1]{#3}}% Store image in box
  \leavevmode\rlap{\usebox1}% Print image
  \rlap{\hspace*{45pt}\vspace*{12pt}\raisebox{\dimexpr\ht1-7.0\baselineskip}{#2}}% Print label
  \phantom{\usebox1}
}
\begin{figure*}[hbt]
\begin{tabular}{@{}p{0.95\linewidth}@{\quad\quad}p{}@{}}
\centering
   \subfigimg[width=0.45\linewidth]{(A)}{{./Figures/H_chain_fit_model_length1.5_tUs_inset}.eps}
   \subfigimg[width=0.45\linewidth]{(B)}{{./Figures/H_chain_fit_model_length2.25_tUs_inset}.eps}\\
 \end{tabular}
\caption{Reconstructed 
model energy versus DMC energy for the H$_{10}$ chain at (A) 1.5 \AA \: and (B) 2.25 \AA \:. The energy range of excitations 
narrows significantly for larger interatomic separation. Insets show the intrinsic atomic orbitals which constitute the one-body space 
which was used for calculating the reduced density matrices (descriptors).  
%\lucas{Need A,B,C labels. $R^2$ should be in the next figure with other vs $r$ things - Done, Huihuo+K}. 
%\lucas{Should standardize these graphs as $E[\Psi]$ and $E_{eff}[\Psi]$ - Done, K}
%  \BDB{why is the energy scale larger than the spread of the data? - Showing it this way lets us use the same axes for both plots (A) and (B), so it's easy to see how the energy spread contracts at longer bond lengths. It also makes it aesthetically easier to show all the annotations and the inset without clipping anything, K} 
%\HHZ{Check whether you mislabelled x and y. Normally, $E[\Psi]$ means the ab-initio energy whereas $E_{eff}[\Psi]$ means the model energy. We usually put ab-initio energy to the x-axis. - Done. Nothing was mislabeled; I've just always put ab-initio E on the y, not the x. I flipped it, K}
\label{fig:fit_quality}
  }
\end{figure*}


Figure~\ref{fig:fit_quality} shows the fitting results of the energy functional $E[\Psi]$ within the sampled $\mathcal LE$ for two representative distances (1.5 and 2.25\AA). As we can see, the model $E_{eff}[\Psi]$ reproduces the \textit{ab initio} $E[\Psi]$ up to certain error that decreases with atomic separation. That is, the fitted Hubbard model provides a more accurate description as separation distance increases, and the system becomes more atomic-like. 

Figure~\ref{fig:Parameters-vs-Bond-t} shows the fitted values of the downfolding paramaters $t$ and $U/t$ at various distances. 
$t$ decreases as the interatomic distance increases, and the value of $U/t$ increases. The single-band Hubbard model qualitatively captures how the system approaches the atomic limit, in which $t$ becomes zero. 

%\lucas{What is the conclusion of this paragraph? - Paragraph broken up and reworded, Huihuo+K}
 
\renewcommand{\subfigimg}[3][,]{%
  \setbox1=\hbox{\includegraphics[#1]{#3}}% Store image in box
  \leavevmode\rlap{\usebox1}% Print image
  \rlap{\hspace*{30pt}\vspace*{20pt}\raisebox{\dimexpr\ht1-5.0\baselineskip}{#2}}% Print label
  \phantom{\usebox1}
}
\begin{figure}[hbt]
   \centering
 \begin{tabular}{@{}p{0.90\linewidth}@{}p{\linewidth}@{}}
   \centering
    \subfigimg[width=0.31\linewidth]{(A)}{./Figures/fitted_t_values_no_offset_h10_chain.eps}
    \subfigimg[width=0.31\linewidth]{(B)}{./Figures/Ust_ratio_vs_separation_h_chain.eps}
    \subfigimg[width=0.32\linewidth]{(C)}{{./Figures/r2_ut_vs_separation_h_chain}.eps}
 \end{tabular}
\caption{ (A) The one-body hopping $t$ parameter as a function of interatomic distance for the periodic H$_{10}$ chain, obtained from a fitted $U$-$t$ model. $t$ declines to zero as $r$ increases. 
(B) The ratio $U/t$ for the fitted parameter values as a function of interatomic separation. The ratio is small at lower bond-lengths, where $t$ is more relevant in describing the system, and larger at longer bond-lengths, where inter-site hopping is less significant. 
(C) The R$^2$ fit parameters obtained from fitting the $U$-$t$ model to the H$_{10}$ chain, as a function of interatomic separation. }\label{fig:Parameters-vs-Bond-t}
\end{figure}

The R$^2$ values obtained from fitting the descriptors to the \textit{ab-initio} energy [see Figure~\ref{fig:Parameters-vs-Bond-t}(C)] also show that the single-band Hubbard model is a good description of the system at large distances, but not at small distances. 
This is primarily because the dynamics of other degrees of freedom (e.g. $2s$ and $2p$ orbitals) become important to the low energy spectrum at small distances. Other interaction terms beyond the on-site Hubbard $U$, such as nearest-neighbor Coulomb interactions and Heisenberg coupling, can also become significant. 
Without including higher orbitals or additional many-body interaction terms, the model gives rise to an incorrect insulator state at small distances. 
Conversely, at larger separations ($r>1.8$\AA), where the system is in an insulator phase \cite{Stella2011}, the model provides a better description. 

%\lucas{this paragraph does not have a clear point. Separate your ideas more. What are you trying to say?} 
%\HHZ{I rewrote this paragraph, maybe Kiel could help to check. - Read over revised paragraph and made some small additional revisions - K}
