\documentclass[prl,12pt,onecolumn,nofootinbib,notitlepage,english,superscriptaddress]{revtex4-1}
\renewcommand{\rmdefault}{cmr}
\usepackage[T1]{fontenc}
\usepackage[latin9]{inputenc}
\setcounter{secnumdepth}{2}
\setcounter{tocdepth}{2}
\usepackage{color}
\usepackage{babel}
\usepackage{latexsym}
\usepackage{float}
\usepackage{amsmath}
\usepackage{amsfonts}
\usepackage{graphicx}
\usepackage{times}   %% Times Roman font
\usepackage{esint}
\usepackage{subfigure}
\usepackage{verbatim}
\usepackage{footmisc}
\usepackage[unicode=true,pdfusetitle,
 bookmarks=false,colorlinks=true,citecolor=blue,urlcolor=blue,linkcolor=red]{hyperref}
\makeatletter
%%%%%%%%%%%%%%%%%%%%%%%%%%%%%% LyX specific LaTeX commands.
\special{papersize=\the\paperwidth,\the\paperheight}

%%%%%%%%%%%%%%%%%%%%%%%%%%%%%% Textclass specific LaTeX commands.
\@ifundefined{textcolor}{}
{%
 \definecolor{BLACK}{gray}{0}
 \definecolor{WHITE}{gray}{1}
 \definecolor{RED}{rgb}{1,0,0}
 \definecolor{GREEN}{rgb}{0,1,0}
 \definecolor{BLUE}{rgb}{0,0,1}
 \definecolor{CYAN}{cmyk}{1,0,0,0}
 \definecolor{MAGENTA}{cmyk}{0,1,0,0}
 \definecolor{YELLOW}{cmyk}{0,0,1,0}
}

\@ifundefined{date}{}{\date{}}
\AtBeginDocument{
  \def\labelitemi{\(\rhd\)}
}
\makeatother

\setlength{\belowcaptionskip}{-7pt}
\newcommand{\SAVE}[1]{}
\newcommand{\HJC}[1]{{\color{RED}{\bf HJC: #1}}}
\newcommand{\prlsec}[1]{\emph{#1---}}
\newcommand{\Ncal}{{\mathcal N}}
\newcommand{\T}{{\mathbf{T}}}
\newcommand{\Jbq}{{J_{bq}}}
\newcommand{\DK}[1]{{\color{BLUE}{\bf DK: #1}}}

\begin{document}
\renewcommand{\thefootnote}{\fnsymbol{footnote}}
\renewcommand\abstractname{}
\title{From real materials to model Hamiltonians: multiscale modelling of strongly correlated electronic systems 
       with many body wavefunctions}

\author{Hitesh J. Changlani}
\affiliation{Department of Physics and Astronomy, Johns Hopkins University, Baltimore, Maryland 21218, USA}
\author{Brian Busemyer}
\affiliation{Department of Physics and Institute for Condensed Matter Theory, University of Illinois at Urbana-Champaign, 
1110 West Green St, Urbana IL 61801, USA}
\author{Kiel Williams}
\affiliation{Department of Physics and Institute for Condensed Matter Theory, University of Illinois at Urbana-Champaign, 
1110 West Green St, Urbana IL 61801, USA}
\author{Lucas K. Wagner}
\affiliation{Department of Physics and Institute for Condensed Matter Theory, University of Illinois at Urbana-Champaign, 
1110 West Green St, Urbana IL 61801, USA}
\date{\today}
\maketitle

\textbf{
Given a realistic material with all its intrinsic complications, how does one develop simple reliable 
models for understanding its properties? Theoretical insight has been the key driver of 
this process leading to simple few-band pictures. When the interactions are comparable or much stronger than 
the kinetic energy, it is convenient to adopt the real space lattice approach. The 
effective Hamiltonian involves a considerable renormalization of parameters. Estimating these values 
is important for several applications in physics and chemistry. Bands/states far from the 
Fermi level play a significant role in the screening of Coulomb interactions leading to Hubbard U's 
that have been traditionally hard to determine reliably. Here we discuss an approach that we 
have been developing which builds in several safeguards and provides a way of obtaining effective Hamiltonians 
which can be simulated using techniques specifically defined for small local Hilbert spaces.
}

\section{Introduction to downfolding}
Survey of DFT based downfolding and known limitations....
Survey of model Hamiltonians - Hubbard, Kanamori, .....

\section{Criteria for Hamiltonian matching}

\section{Posing the problem/Visualizing the goodness of fit/connecting to tools from data science}

\section{Choice of norm and avoiding overfitting: $L_1$, $L_2$ or something else?}

\section{Choosing the optimal one body space/ Dual optimization of Hamiltonian parameters and basis sets}

\section{Simple examples - lattice to lattice downfolding - Three to One band model and the "effective" d orbitals}
Consider the three band Hubbard model.... 

\section{Realistic applications from one to many band models- silicon, carbon, transition metals, transition metal oxides}

\section{The need for including spin-orbit terms, need for QMC for this case}

\section{Prospects for other areas: Magnetism, small energy scales}

\section{Applications to non QMC methods - coupled cluster, FCI, HCI}

\section{Strengths and limitations of effective Hamiltonian approach}

\end{document}
