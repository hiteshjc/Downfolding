\documentclass[prl,12pt,onecolumn,nofootinbib,notitlepage,english,superscriptaddress]{revtex4-1}
\renewcommand{\rmdefault}{cmr}
\usepackage[T1]{fontenc}
\usepackage[latin9]{inputenc}
\setcounter{secnumdepth}{2}
\setcounter{tocdepth}{2}
\usepackage{color}
\usepackage{babel}
\usepackage{latexsym}
\usepackage{float}
\usepackage{amsmath}
\usepackage{amsfonts}
\usepackage{graphicx}
\usepackage{times}   %% Times Roman font
\usepackage{esint}
\usepackage{subfigure}
\usepackage{verbatim}
\usepackage{footmisc}
\usepackage[unicode=true,pdfusetitle,
 bookmarks=false,colorlinks=true,citecolor=blue,urlcolor=blue,linkcolor=red]{hyperref}
\makeatletter
%%%%%%%%%%%%%%%%%%%%%%%%%%%%%% LyX specific LaTeX commands.
\special{papersize=\the\paperwidth,\the\paperheight}

%%%%%%%%%%%%%%%%%%%%%%%%%%%%%% Textclass specific LaTeX commands.
\@ifundefined{textcolor}{}
{%
 \definecolor{BLACK}{gray}{0}
 \definecolor{WHITE}{gray}{1}
 \definecolor{RED}{rgb}{1,0,0}
 \definecolor{GREEN}{rgb}{0,1,0}
 \definecolor{BLUE}{rgb}{0,0,1}
 \definecolor{CYAN}{cmyk}{1,0,0,0}
 \definecolor{MAGENTA}{cmyk}{0,1,0,0}
 \definecolor{YELLOW}{cmyk}{0,0,1,0}
}

\@ifundefined{date}{}{\date{}}
\AtBeginDocument{
  \def\labelitemi{\(\rhd\)}
}
\makeatother

\setlength{\belowcaptionskip}{-7pt}
\newcommand{\SAVE}[1]{}
\newcommand{\HJC}[1]{{\color{RED}{\bf HJC: #1}}}
\newcommand{\prlsec}[1]{\emph{#1---}}
\newcommand{\Ncal}{{\mathcal N}}
\newcommand{\T}{{\mathbf{T}}}
\newcommand{\Jbq}{{J_{bq}}}
\newcommand{\DK}[1]{{\color{BLUE}{\bf DK: #1}}}

\begin{document}
\renewcommand{\thefootnote}{\fnsymbol{footnote}}
\renewcommand\abstractname{}
\title{From real materials to model Hamiltonians: multiscale modelling of strongly correlated electronic systems 
       with many body wavefunctions}

\author{Hitesh J. Changlani}
\affiliation{Department of Physics and Astronomy, Johns Hopkins University, Baltimore, Maryland 21218, USA}
\author{Brian Busemyer}
\affiliation{Department of Physics and Institute for Condensed Matter Theory, University of Illinois at Urbana-Champaign, 
1110 West Green St, Urbana IL 61801, USA}
\author{Kiel Williams}
\affiliation{Department of Physics and Institute for Condensed Matter Theory, University of Illinois at Urbana-Champaign, 
1110 West Green St, Urbana IL 61801, USA}
\author{Lucas K. Wagner}
\affiliation{Department of Physics and Institute for Condensed Matter Theory, University of Illinois at Urbana-Champaign, 
1110 West Green St, Urbana IL 61801, USA}
\date{\today}
\maketitle

\textbf{
Given a realistic material with all its intrinsic complications, how does one develop simple reliable 
models for understanding its properties? Theoretical insight has been the key driver of 
this process leading to simple few-band pictures. When the interactions are comparable or much stronger than 
the kinetic energy, it is convenient to adopt the real space lattice approach. The 
effective Hamiltonian involves a considerable renormalization of parameters. Estimating these values 
is important for several applications in physics and chemistry. Bands/states far from the 
Fermi level play a significant role in the screening of Coulomb interactions leading to Hubbard U's 
that have been traditionally hard to determine reliably. Here we discuss an approach that we 
have been developing which builds in several safeguards and provides a way of obtaining effective Hamiltonians 
which can be simulated using techniques specifically defined for small local Hilbert spaces.
}

\section{Introduction to downfolding}

\HJC{Survey of DFT based downfolding and known limitations....Survey of model Hamiltonians - Hubbard, Kanamori, .....}
\HJC{Modify intro - this is from the paper........ }

Reliably simulating quantum systems remains a challenge 
for the physics and chemistry communities. Calculations on real materials 
are primarily carried out with density functional theory (DFT), 
which in principle is exact, but in practice is limited by the quality of the exchange functional. 
On the other hand are approximate model Hamiltonians (such as the Hubbard model) 
which describe the low energy physics solely in terms of the valence electrons and 
are crucial to our understanding of physical phenomena such 
as antiferromagnetism and high temperature superconductivity. 
However, transitioning from actual materials to an effective Hamiltonian on a lattice relies 
on physical insight and/or fits to experimental data, and this notion is not always rigorously well justified.
%~\footnote{
%While terms introduced in a lattice model are often very well motivated by the 
%underlying physics and the symmetries of the problem, 
%the magnitude (and sometimes even sign) the couplings is unknown.
 
This effective reduction of the complexity of the calculation is known as "downfolding" 
and is commonly carried out using DFT-based approaches. 
These methods use the electronic band structure to generate a tight binding model, 
but are generally silent on the role of interactions (i.e. two and higher body terms). 
Incorrect estimation of the strength of the interactions 
in the case of systems with strong electron-electron correlations leads to 
outcomes in clear violations of experiments. For example, the local density 
approximation (LDA) predicts undoped LaCu$_2$O$_4$ 
to be metallic instead of insulating, requiring adhoc inputs for 
the downfolding procedure. Other schemes for magnetic systems involve fitting DFT energies for 
various spin configurations to effective Heisenberg and bilinear biquadratic models.    

Once an effective Hamiltonian in the \emph{reduced} Hilbert space 
is obtained, it can be used to perform a lattice-model calculation 
on a \emph{much} bigger system. Many techniques such as the density matrix 
renormalization group (DMRG), tensor networks, 
dynamical field theory (DMFT), density matrix embedding (DMET) 
and lattice quantum Monte Carlo (QMC) methods
have been developed to deal with such model Hamiltonians. 
This multi-step modeling procedure is needed since the \emph{ab-initio} 
calculations for a given system size are, in general, much more 
computationally expensive than the equivalent lattice calculations. 
Large sizes are crucial to study finite size effects, and in turn 
theoretically establish the presence of a phase. For example, in the case of 
frustrated magnetic systems, many nearly-degenerate states exist some of which can be 
stabilized only on very large unit cells. 
In addition, excited states and dynamical correlation functions have traditionally 
been difficult in \emph{ab-initio} approaches, 
but have seen progress for lattice model methods.

Various efforts have been taken to develop accurate downfolding 
methods. For example, Aryasetiawan et al.
have laid out a post-DFT framework for obtaining 
static and frequency dependent Hamiltonians using the 
constrained-RPA approach. While the method has been applied to 
a wide range of systems, 
it is only recently that its accuracy is being rigorously checked. 
Other approaches include traditional Lowdin downfolding 
within a stochastic approach and 
canonical transformation theory, 
also being actively pursued as quantum chemistry 
methods
%~\footnote{A special case of canonical transformation theory is 
%the Schrieffer-Wolf transformation used in the 
%BCS theory of superconductivity}.  

In this paper, we apply the \emph{ab-initio} 
Quantum Monte Carlo (QMC) approach~\cite{Ceperley_Alder,Foulkes_review}, 
one of the most accurate and highly scalable wavefunction based electronic-structure 
methods. The QMC method works directly in the continuum and 
explicitly introduces correlation into the Slater determinant obtained 
from DFT or Hartree Fock calculations. For energies of eigenstates, obtained 
by stochastic projection, the only systematic error is in the assumption of the 
nodes of the many-body wavefunction: in practice, 
this approximation is estimated to miss $ 5 - 10 \%$ 
of the correlation energy. Correlation functions and density matrices are generally 
less accurate, but good wavefunctions can bring down the errors significantly. 
For arbitrary wavefunctions which can be evaluated in polynomial time, 
the \emph{only} error is purely statistical and 
can be brought down to arbitrarily small values by increased sampling.

While all results presented here utilize data generated from 
\emph{ab-initio} QMC, the idea of fitting data from many-body wavefunctions to models applies 
to other methods as well. Information from many-body wavefunctions detects 
inadequacies of a proposed model: the deviations suggest the form of additional missing terms. 
Crucially, a wavefunction-based method does not differentiate between kinetic 
and potential energy terms: all terms are treated the same way. 
The approach is complementary to semi-empirical methods that 
fit models solely to available energy spectra. 


\section{Criteria for Hamiltonian matching}

\HJC{Visualizing the goodness of fit/connecting to tools from data science}
\HJC{Choice of norm and avoiding overfitting: $L_1$, $L_2$ or something else?}

\section{Simple examples}
\HJC{Choosing the optimal one body space/ Dual optimization of Hamiltonian parameters and basis sets}
\HJC{Simple examples - lattice to lattice downfolding - Three to One band model and the "effective" d orbitals}

\section{Realistic applications - silicon, carbon, transition metals, transition metal oxides}

\section{Prospects}
\HJC{The need for including spin-orbit terms, need for QMC for this case}
\HJC{Other areas: Magnetism, small energy scales}
\HJC{Applications to non QMC methods - coupled cluster, FCI, HCI}
\HJC{Strengths and limitations of effective Hamiltonian approach}

\bibliographystyle{prsty}
\bibliography{refs}

\end{document}
