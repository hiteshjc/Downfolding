\documentclass[prl,12pt,onecolumn,nofootinbib,notitlepage,english,superscriptaddress]{revtex4-1}
\renewcommand{\rmdefault}{cmr}
\usepackage[T1]{fontenc}
\usepackage[latin9]{inputenc}
\setcounter{secnumdepth}{2}
\setcounter{tocdepth}{2}
\usepackage{color}
\usepackage{babel}
\usepackage{latexsym}
\usepackage{float}
\usepackage{amsmath}
\usepackage{amsfonts}
\usepackage{graphicx}
\usepackage{times}   %% Times Roman font
\usepackage{esint}
\usepackage{subfigure}
\usepackage{verbatim}
\usepackage{braket}
\usepackage{footmisc}
\usepackage[unicode=true,pdfusetitle,
 bookmarks=false,colorlinks=true,citecolor=blue,urlcolor=blue,linkcolor=red]{hyperref}
\makeatletter
%%%%%%%%%%%%%%%%%%%%%%%%%%%%%% LyX specific LaTeX commands.
\special{papersize=\the\paperwidth,\the\paperheight}

%%%%%%%%%%%%%%%%%%%%%%%%%%%%%% Textclass specific LaTeX commands.
\@ifundefined{textcolor}{}
{%
 \definecolor{BLACK}{gray}{0}
 \definecolor{WHITE}{gray}{1}
 \definecolor{RED}{rgb}{1,0,0}
 \definecolor{GREEN}{rgb}{0,0.6,0}
 \definecolor{BLUE}{rgb}{0,0,1}
 \definecolor{CYAN}{cmyk}{1,0,0,0}
 \definecolor{MAGENTA}{cmyk}{0,1,0,0}
 \definecolor{YELLOW}{cmyk}{0,0,1,0}
}

\@ifundefined{date}{}{\date{}}
\AtBeginDocument{
  \def\labelitemi{\(\rhd\)}
}
\makeatother

\setlength{\belowcaptionskip}{-7pt}
\newcommand{\SAVE}[1]{}
\newcommand{\HJC}[1]{{\color{RED}{\bf HJC: #1}}}
\newcommand{\lucas}[1]{{\color{GREEN}{\bf LKW: #1}}}

\newcommand{\prlsec}[1]{\emph{#1---}}
\newcommand{\Ncal}{{\mathcal N}}
\newcommand{\T}{{\mathbf{T}}}
\newcommand{\Jbq}{{J_{bq}}}
\newcommand{\DK}[1]{{\color{BLUE}{\bf DK: #1}}}

\begin{document}
\renewcommand{\thefootnote}{\fnsymbol{footnote}}
\renewcommand\abstractname{}
\title{From real materials to model lattice Hamiltonians: multi-scale modelling of strongly correlated electronic systems 
       with information from many body wavefunctions}

\maketitle




\section{Introduction to downfolding the many electron problem}
\begin{itemize}
\item Why we want interacting models
	\begin{itemize}
	\item Coarse-graining to get to longer length/time scales
	\item Conceptual understanding 
	\item \lucas{other thoughts?} 
	\end{itemize}
\item Standard methods. DFT-based downfolding, CRPA-based downfolding. \lucas{What is the issue with these methods? Accuracy? Being able to tell if a model is good or not? This will determine what we talk about in the paper. When are these methods good?} 
\item We are presenting a more general solution to this problem:
   \begin{itemize}
      \item  can tell us when a model is good/bad, how much.
      \item  can take advantage of many-body wave functions
    \end{itemize}
\end{itemize}

\section{Theory: compression of the energy functional}


 
\begin{itemize}
\item What principles are important for connecting a problem of many electrons to a problem of few electrons
 \begin{itemize}
\item Analogy to fitting classical PES
    \begin{itemize}
    \item Some modes get gapped out (bond stretch in water, for example)
    \end{itemize}
\item $E[\Psi]$ is like the classical PES. High energy directions also get gapped out. 
\item A downfolded or coarse-grained model is a representation of the low-energy $E[\Psi]$ in terms of descriptors $\{d_i[\Psi]\}$
\item For many systems, when we choose $d_i$ to be expectation values of density matrices, $E[\Psi]$ is a linear function.
\end{itemize}


	
\item Protocol 
   \begin{itemize}
   \item Generate a set of potential descriptors consistent with symmetry and physical knowledge
   \item Generate a set of wave functions putatively in the low energy space $|\Psi_i\rangle \in {\cal LE} $, and evaluate their descriptor and energy expectation values.
   \item Assessment point: 
      \begin{itemize}
      \item If two descriptors are highly correlated in the samples, either the samples are not sufficient or the descriptor space is overcomplete.
      \item If two wave functions have identical or near-identical descriptor sets, then either take the lowest energy (these are approximations to the low energy manifold and the lowest will be the best approximation), or consider expanding the descriptor set.
      \end{itemize}
   \item Fitting: 
      	\begin{itemize}
      	\item Choice of norm and avoiding over-fitting: $L_1$, $L_2$ or something else? 
      	\item What does the A matrix encode - standard tricks of fitting...
	\end{itemize}

   \item Iteration: if we can solve the resultant model, does it predict low-lying states that are not in our sample set? Concept of intruder states; one example is simple stretched N$_2$ models.
   \end{itemize}

\item Subtleties: Choice of states, issues related to 1 particle basis...
	\begin{itemize}
	\item choice of energy window, data selection, many body states
	\item Static vs dynamic
	\item 1 particle basis, localized orbitals
	\item Specific algorithmic details?
	\end{itemize}

\end{itemize}


\section{Representative Examples}
\subsection{Three-band Hubbard model to one band Hubbard model at half filling}
\begin{itemize}
	\item What is the 3 band model? Why the 3 band model.. relevance etc?
	\HJC{Hitesh working on this. Example of lattice to lattice downfolding and a concrete SIMPLE example of what it means to downfold. Performing transferability calculation (8+16 site) and more analyses with existing data}
	\begin{itemize}
		\item VERY Brief history and relevance to cuprates	
		\item Simple example of lattice to lattice downfolding, Explain model and parameter set \HJC{In response to LKW: Not worried about whether parameters correspond to cuprate or not... This paper is to explain downfolding... thus I have considered a model only with $t_{pd}$, $\Delta$ and $U_d$ and map it to a 1 band Hubbard with $t$ and $U$. For pedagogy I have scanned $U_d$ and $\Delta$. I will not use any parameter set you gave me.}
	\end{itemize}
	\item Downfolding 3 band model
	\begin{itemize}
		\item Model we want to map it to ... 
		\item Effective one particle orbitals and "unitary"
		\item Effective interactions
	\end{itemize}
\item Results and Discussion
	\begin{itemize}
		\item U/t and t with Ud/tpd. 
		\item Variation with Ep-Ed
		\item Hybridization parameter
		\item Transferability?
	\end{itemize}
\end{itemize}

\subsection{Hydrogen Chain}
\HJC{Huihuo working on this. Extension to simple ab initio problem. Issue with transferability being resolved}
\lucas{I thought that Kiel was also working on this for long-range V} 
\begin{itemize}
	\item Why hydrogen chain ? One of the simplest ab initio systems... 
	\begin{itemize}
		\item VERY brief intro	
		\item Simplest ab initio system 
	\end{itemize}
	\item Downfolding it
	\begin{itemize}
		\item Models we want to map it to ... 
		\item Kohn Sham - Wannier
		\item Effective interactions
	\end{itemize}
\item Results and Discussion
	\begin{itemize}
		\item U/t and t  
		\item Transferability
	\end{itemize}
\end{itemize}

\subsection{Graphene and hydrogen honeycomb lattice (SHORT section)}
\HJC{Huihuo + Hitesh.. Role of screening from other orbitals etc. Highlights importance of core and virtual orbitals. More or less in place...}
\begin{itemize}
	\item Other realistic example......... Role of sigma and pi electrons
	\begin{itemize}
		\item Set up of graphene and hydrogen-honeycomb (Why this choice)
		\item short range Models we want to map it to ... 
		\item Role of sigma electrons in screening 
		\item Compare the two to highlight role... 
	\end{itemize}
\end{itemize}



\section{Future Prospects and need for more work in the area}
\HJC{Other areas: Magnetism, small energy scales}
\lucas{We already do magnetism with this method, right? It might be worth mentioning that it justifies the method of looking at the Neel state versus the AFM state.} 
\HJC{Applications to non QMC methods - coupled cluster, FCI, HCI}
\HJC{Strengths and limitations of effective Hamiltonian approach}
\begin{itemize}
	\item Strengths and limitations of approach - separation of energy scales etc..
	\item Other places where it can be applied
	\item Non QMC methods
\end{itemize}

\bibliographystyle{unsrt}
\bibliography{refs}

\end{document}
