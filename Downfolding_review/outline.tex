\documentclass[prl,12pt,onecolumn,nofootinbib,notitlepage,english,superscriptaddress]{revtex4-1}
\renewcommand{\rmdefault}{cmr}
\usepackage[T1]{fontenc}
\usepackage[latin9]{inputenc}
\setcounter{secnumdepth}{2}
\setcounter{tocdepth}{2}
\usepackage{color}
\usepackage{babel}
\usepackage{latexsym}
\usepackage{float}
\usepackage{amsmath}
\usepackage{amsfonts}
\usepackage{graphicx}
\usepackage{times}   %% Times Roman font
\usepackage{esint}
\usepackage{subfigure}
\usepackage{verbatim}
\usepackage{braket}
\usepackage{footmisc}
\usepackage[unicode=true,pdfusetitle,
 bookmarks=false,colorlinks=true,citecolor=blue,urlcolor=blue,linkcolor=red]{hyperref}
\makeatletter
%%%%%%%%%%%%%%%%%%%%%%%%%%%%%% LyX specific LaTeX commands.
\special{papersize=\the\paperwidth,\the\paperheight}

%%%%%%%%%%%%%%%%%%%%%%%%%%%%%% Textclass specific LaTeX commands.
\@ifundefined{textcolor}{}
{%
 \definecolor{BLACK}{gray}{0}
 \definecolor{WHITE}{gray}{1}
 \definecolor{RED}{rgb}{1,0,0}
 \definecolor{GREEN}{rgb}{0,0.6,0}
 \definecolor{BLUE}{rgb}{0,0,1}
 \definecolor{CYAN}{cmyk}{1,0,0,0}
 \definecolor{MAGENTA}{cmyk}{0,1,0,0}
 \definecolor{YELLOW}{cmyk}{0,0,1,0}
}

\@ifundefined{date}{}{\date{}}
\AtBeginDocument{
  \def\labelitemi{\(\rhd\)}
}
\makeatother

\setlength{\belowcaptionskip}{-7pt}
\newcommand{\SAVE}[1]{}
\newcommand{\HJC}[1]{{\color{RED}{\bf HJC: #1}}}
\newcommand{\lucas}[1]{{\color{GREEN}{\bf LKW: #1}}}

\newcommand{\prlsec}[1]{\emph{#1---}}
\newcommand{\Ncal}{{\mathcal N}}
\newcommand{\T}{{\mathbf{T}}}
\newcommand{\Jbq}{{J_{bq}}}
\newcommand{\DK}[1]{{\color{BLUE}{\bf DK: #1}}}

\begin{document}
\renewcommand{\thefootnote}{\fnsymbol{footnote}}
\renewcommand\abstractname{}
\title{From real materials to model lattice Hamiltonians: multi-scale modelling of strongly correlated electronic systems 
       with information from many body wavefunctions}

\maketitle




\section{Introduction to downfolding the many electron problem}
\begin{itemize}
\item Why we want interacting models
	\begin{itemize}
	\item Coarse-graining to get to longer length/time scales
	\item Conceptual understanding 
	\item \lucas{other thoughts?} 
	\end{itemize}
\item Standard methods. DFT-based downfolding, CRPA-based downfolding. \lucas{What is the issue with these methods? Accuracy? Being able to tell if a model is good or not? This will determine what we talk about in the paper. When are these methods good?} 
\item We are presenting a more general solution to this problem:
   \begin{itemize}
      \item  can tell us when a model is good/bad, how much.
      \item  can take advantage of many-body wave functions
    \end{itemize}
\end{itemize}

\section{Theory: compression of the energy functional}

\begin{itemize}
\item Analogy to fitting classical PES
    \begin{itemize}
    \item Some modes get gapped out (bond stretch in water, for example)
    \end{itemize}
\item $E[\Psi]$ is like the classical PES. High energy bits also get gapped out. 
\item A downfolded or coarse-grained model is a representation of the low-energy $E[\Psi]$ in terms of descriptors $\{d_i[\Psi]\}$
\item For many systems, when we choose $d_i$ to be expectation values of density matrices, $E[\Psi]$ is a linear function.
\end{itemize}

 
\section{Criteria for Hamiltonian matching}
\HJC{Discuss what metrics are being matched. Give details} 
\lucas{Why do we choose these criteria? What are the advantages of these criteria versus say matching the spectrum? }

\subsection{Choice of states, issues related to 1 particle basis...}
\HJC{Algorithms of state selection, localized orbitals...} 

\subsection{Fitting procedure: $Ax=E$ method}
\HJC{Choice of norm and avoiding over-fitting: $L_1$, $L_2$ or something else?}

\section{Representative Examples}
\subsection{Three-band Hubbard model to one band Hubbard model at half filling}
\HJC{Hitesh working on this. Example of lattice to lattice downfolding and a concrete SIMPLE example of what it means to downfold. Performing transferability calculation (8+16 site) and more analyses with existing data}

\subsection{Hydrogen Chain}
\HJC{Huihuo working on this. Extension to simple ab initio problem. Issue with transferability being resolved}

\lucas{I thought that Kiel was also working on this for long-range V} 

\subsection{Graphene and hydrogen honeycomb lattice}
\HJC{Huihuo + Hitesh.. Role of screening from other oribtals etc. Highlights importance of core and virtual orbitals. More or less in place...}

\section{Future Realistic applications - silicon OR carbon OR transition metals OR transition metal oxides??}
\HJC{Optional section ?? Depends on Kiel and Brian... currently no consensus... Wait 2 weeks (end of Sept, 2017)} 

\lucas{We will not have anything useful for TMOs here. It might be worth trying to derive a tight-binding model for Silicon. I did start this calculation if anyone would like to continue it.} 

\section{Future Prospects and need for more work in the area}
\HJC{The need for including spin-orbit terms, need for QMC for this case}
\lucas{Isn't this beyond the scope of this article? } 

\HJC{Other areas: Magnetism, small energy scales}
\lucas{We already do magnetism with this method, right? It might be worth mentioning that it justifies the method of looking at the Neel state versus the AFM state.} 
\HJC{Applications to non QMC methods - coupled cluster, FCI, HCI}

\HJC{Strengths and limitations of effective Hamiltonian approach}


\bibliographystyle{unsrt}
\bibliography{refs}

\end{document}
