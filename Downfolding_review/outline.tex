\documentclass[prl,12pt,onecolumn,nofootinbib,notitlepage,english,superscriptaddress]{revtex4-1}
\renewcommand{\rmdefault}{cmr}
\usepackage[T1]{fontenc}
\usepackage[latin9]{inputenc}
\setcounter{secnumdepth}{2}
\setcounter{tocdepth}{2}
\usepackage{color}
\usepackage{babel}
\usepackage{latexsym}
\usepackage{float}
\usepackage{amsmath}
\usepackage{amsfonts}
\usepackage{graphicx}
\usepackage{times}   %% Times Roman font
\usepackage{esint}
\usepackage{subfigure}
\usepackage{verbatim}
\usepackage{braket}
\usepackage{footmisc}
\usepackage[unicode=true,pdfusetitle,
 bookmarks=false,colorlinks=true,citecolor=blue,urlcolor=blue,linkcolor=red]{hyperref}
\makeatletter
%%%%%%%%%%%%%%%%%%%%%%%%%%%%%% LyX specific LaTeX commands.
\special{papersize=\the\paperwidth,\the\paperheight}

%%%%%%%%%%%%%%%%%%%%%%%%%%%%%% Textclass specific LaTeX commands.
\@ifundefined{textcolor}{}
{%
 \definecolor{BLACK}{gray}{0}
 \definecolor{WHITE}{gray}{1}
 \definecolor{RED}{rgb}{1,0,0}
 \definecolor{GREEN}{rgb}{0,0.6,0}
 \definecolor{BLUE}{rgb}{0,0,1}
 \definecolor{CYAN}{cmyk}{1,0,0,0}
 \definecolor{MAGENTA}{cmyk}{0,1,0,0}
 \definecolor{YELLOW}{cmyk}{0,0,1,0}
}

\@ifundefined{date}{}{\date{}}
\AtBeginDocument{
  \def\labelitemi{\(\rhd\)}
}
\makeatother

\setlength{\belowcaptionskip}{-7pt}
\newcommand{\SAVE}[1]{}
\newcommand{\HJC}[1]{{\color{RED}{\bf HJC: #1}}}
\newcommand{\lucas}[1]{{\color{GREEN}{\bf LKW: #1}}}

\newcommand{\prlsec}[1]{\emph{#1---}}
\newcommand{\Ncal}{{\mathcal N}}
\newcommand{\T}{{\mathbf{T}}}
\newcommand{\Jbq}{{J_{bq}}}
\newcommand{\DK}[1]{{\color{BLUE}{\bf DK: #1}}}

\begin{document}
\renewcommand{\thefootnote}{\fnsymbol{footnote}}
\renewcommand\abstractname{}
\title{From real materials to model lattice Hamiltonians: multi-scale modelling of strongly correlated electronic systems 
       with information from many body wavefunctions}

\author{Hitesh J. Changlani}
\affiliation{Department of Physics and Astronomy, Johns Hopkins University, Baltimore, Maryland 21218, USA}
\author{Huihuo Zheng}
\affiliation{
Argonne Leadership Computing Facility, Argonne National Laboratory, 9700 South Cass Avenue, Lemont, 60439, Illinois, USA}
\author{Brian Busemeyer}
\affiliation{Department of Physics and Institute for Condensed Matter Theory, University of Illinois at Urbana-Champaign, 
1110 West Green St, Urbana IL 61801, USA}
\author{Kiel Williams}
\affiliation{Department of Physics and Institute for Condensed Matter Theory, University of Illinois at Urbana-Champaign, 
1110 West Green St, Urbana IL 61801, USA}
\author{Lucas K. Wagner}
\affiliation{Department of Physics and Institute for Condensed Matter Theory, University of Illinois at Urbana-Champaign, 
1110 West Green St, Urbana IL 61801, USA}
\date{\today}
\maketitle

\textbf{
Given a realistic material with all its intrinsic complications, how does one develop a 
simple reliable model for understanding its properties? Theoretical insight has been the key driver of 
this process leading to simple few-band pictures. When the interactions are comparable or much stronger than 
the kinetic energy, it is convenient to adopt the real space lattice approach and think in terms of Hubbard or Kanamori type 
Hamiltonians involving only the low energy electrons. 
But this is no easy task, since the effective Hamiltonian involves a considerable renormalization of parameters with respect 
to the bare Coulomb values. While the kinetic energy is dominated by contributions from bands or states energetically 
near the Fermi level, screened interactions depend on states even \emph{far} away from it, leading to Hubbard U's 
that have been traditionally hard to reliably determine. 
Here we discuss an approach that treats 
the kinetic and potential parts of the Hamiltonian democratically and one that provides a transparent way 
of obtaining effective Hamiltonians using data from many body wavefunctions, 
and whose validity can be systematically checked.
We emphasize that determining the effective Hamiltonian reliably is crucial for 
several applications in physics and chemistry not only for quantitative accuracy, but also a 
correct qualitative picture of strongly correlated materials. 
}

\section{Introduction to downfolding the many electron problem}
\HJC{Brief Survey and motivations ....}
\lucas{Need more details here. What is the reason that we're writing this paper? What accuracy should we expect? } 
 
\section{Criteria for Hamiltonian matching}
\HJC{Discuss what metrics are being matched. Give details} 
\lucas{Why do we choose these criteria? What are the advantages of these criteria versus say matching the spectrum? }

\subsection{Choice of states, issues related to 1 particle basis...}
\HJC{Algorithms of state selection, localized orbitals...} 

\subsection{Fitting procedure: $Ax=E$ method}
\HJC{Choice of norm and avoiding over-fitting: $L_1$, $L_2$ or something else?}

\section{Representative Examples}
\subsection{Three-band Hubbard model to one band Hubbard model at half filling}
\HJC{Hitesh working on this. Example of lattice to lattice downfolding and a concrete SIMPLE example of what it means to downfold. Performing transferability calculation (8+16 site) and more analyses with existing data}

\subsection{Hydrogen Chain}
\HJC{Huihuo working on this. Extension to simple ab initio problem. Issue with transferability being resolved}

\lucas{I thought that Kiel was also working on this for long-range V} 

\subsection{Graphene and hydrogen honeycomb lattice}
\HJC{Huihuo + Hitesh.. Role of screening from other oribtals etc. Highlights importance of core and virtual orbitals. More or less in place...}

\section{Future Realistic applications - silicon OR carbon OR transition metals OR transition metal oxides??}
\HJC{Optional section ?? Depends on Kiel and Brian... currently no consensus... Wait 2 weeks (end of Sept, 2017)} 

\lucas{We will not have anything useful for TMOs here. It might be worth trying to derive a tight-binding model for Silicon. I did start this calculation if anyone would like to continue it.} 

\section{Future Prospects and need for more work in the area}
\HJC{The need for including spin-orbit terms, need for QMC for this case}
\lucas{Isn't this beyond the scope of this article? } 

\HJC{Other areas: Magnetism, small energy scales}
\lucas{We already do magnetism with this method, right? It might be worth mentioning that it justifies the method of looking at the Neel state versus the AFM state.} 
\HJC{Applications to non QMC methods - coupled cluster, FCI, HCI}

\HJC{Strengths and limitations of effective Hamiltonian approach}


\bibliographystyle{unsrt}
\bibliography{refs}

\end{document}
