\newtheorem{theorem}{Theorem}
\newtheorem{definition}{Definition}


\section{Downfolding as a compression of the energy functional}

\subsection{Theory} 

Suppose we start with a quantum system with Hamiltonian $H$ and Hilbert space ${\cal H}$.

\begin{definition}
$E[\Psi] = \frac{\bra{\Psi}H\ket{\Psi}}{\braket{\Psi|\Psi}}$
\end{definition}


\begin{definition}
Let ${\cal H}$ be a Hilbert space. Then ${\cal LE}(H,N)$ is a subset of ${\cal H}$ spanned by $N$ vectors given by the lowest energy solutions to $H\ket{\Phi_i}=E_i{\Phi_i}$. 
\end{definition}

\begin{definition}
$H_{eff}$ is an operator on the Hilbert space ${\cal LE(H,N)}$.	\lucas{Does this need to be linear?} 
\end{definition}


\begin{definition}
The effective model $E_{eff}[\Psi]=\frac{\bra{\Psi}H_{eff}\ket{\Psi}}{\braket{\Psi|\Psi}}$ is a functional from ${\cal LE} \rightarrow \mathbb{R}$
\end{definition}



\begin{theorem}
\label{theorem:criticalpoint}
$E[\Psi]$ has a critical point only where $\Psi$ is an eigenstate of $H$.
\end{theorem}
\begin{proof}
This has been proven already by \lucas{find citation} 	
\end{proof}


\begin{theorem} 	
If $E_{eff}[\Psi] = E[\Psi], \forall \ket{\Psi} \in {\cal LE}$, then $H_{eff} \ket{\Phi_i} = E_i\ket{\Phi_i}$ for all eigenstates $\ket{\Phi_i}\in {\cal LE}$.
\end{theorem}
\begin{proof}
	Suppose that $\ket{\Phi_i}$ is an eigenstate of $H$. Then 
	\begin{equation}
	\left.\frac{\partial E[\Psi]}{\partial\Psi}\right|_{\Psi=\Phi_i} = 0 
	\end{equation}
If this is true, then certainly the derivative on the low-energy manifold is also zero.
Since $E_{eff}[\Psi] = E[\Psi]$, 
\begin{equation}
	\left.\frac{\partial E_{eff}[\Psi]}{\partial\Psi}\right|_{\Psi=\Phi_i} = 0 
\end{equation}
for $\ket{\Psi} \in {\cal LE}$. Using Theorem~\ref{theorem:criticalpoint}, $H_{eff} \ket{\Phi_i} = E_i\ket{\Phi_i}$. 
Similarly, if the derivative is not zero, then the wave function must not be an eigenstate.
\end{proof}

We have thus reduced the problem of finding an effective Hamiltonian $H_{eff}$ that reproduces the low-energy spectrum of $H$ to matching the corresponding energy functionals $E[\Psi]$ and $E_{eff}[\Psi]$. 
An important result of this is that it is not necessary to diagonalize either of the Hamiltonians; one must only be able to select wave functions from the low-energy space ${\cal LE}$.
As we shall see, this can be substantially easier than attaining full eigenstates.


The theory presented above maps coarse-graining into a functional approximation problem. 
This is still rather intimidating, since even supposing one can generate wave functions in the low-energy space, they are still complicated objects in a very large space.
An effective way to accomplish this is through the use of descriptors, $d_i[\Psi]$, which map from ${\cal H} \rightarrow \mathbb{R}$.
Then we can approximate the energy functional as follows
\begin{equation}
E_{eff}[\Psi] \simeq \sum_i f_i(d_i[\Psi]),
\end{equation}
where $f_i$ are some parameterized functions.
This will allow us to use data analysis techniques to efficiently describe $E_{eff}$.

\subsection{Practical protocol}

\tikzstyle{decision} = [diamond, draw, fill=blue!10, 
    text width=4.5em, text badly centered, node distance=3cm, inner sep=0pt]
\tikzstyle{block} = [rectangle, draw, fill=blue!10, 
    text width=5em, text centered, rounded corners, minimum height=4em]
\tikzstyle{result} = [rectangle, draw, fill=red!10, 
    text width=5em, text centered, rounded corners, minimum height=4em]
\tikzstyle{line} = [draw,-latex',very thick]
\tikzstyle{cloud} = [draw, ellipse,fill=red!20, node distance=3cm,
    minimum height=2em]
\begin{figure*}
\begin{tikzpicture}[scale=2,node distance = 3cm, auto]
    % Place nodes
    \node [block] (wfs) {Generate $\ket{\Psi_i} \in {\cal LE}$};
    \node [block, right of=wfs] (descriptors) {Generate $d_i[\Psi_i]$ and $E_i[\Psi_i]$};
    \node [block, right of=descriptors] (assess) {Assess descriptors};
    \node [block, right of=assess] (ansatz) {Ansatz: $E_i \simeq \sum_i c_i d_i$};
    \node [block, right of=ansatz] (fit) {Fit bounding plane};
    \node [result, right of=fit] (model) {Effective model};
    % Draw edges
    \path [line] (wfs) -- (descriptors);
    \path [line] (descriptors) -- (assess);
    \path [line] (assess) --  (ansatz);
    \path [line] (ansatz) --  (fit);
    \path [line] (fit) --  (model);

    \path [line] (assess.south) -- ($ (assess.south) + (0,-0.2)$) 
                 -- node [below] {Incomplete sampling} 
                 ($ (wfs.south) + (0,-0.2)$) --  (wfs.south);

\end{tikzpicture}
\caption{A practical protocol for fitting effective models to {\it ab initio} data.}
\label{fig:protocol} 
\end{figure*}

A practical protocol is presented in Fig~\ref{fig:protocol}. 
Let's go through this step by step.

\paragraph{Generating $\ket{\Psi_i}\in {\cal LE}$}. 
\begin{itemize}
\item Ideally sample the whole space; it is probably too big.
\item Dependency on expected model: if you have a $U$ you should vary double occupancy and occupation {\it independently} 
\item If the model is non-interacting, singles excitations are enough; doubles excitations can give information about how good the model is.
\end{itemize}

\paragraph{Generate $d_i[\Psi_i]$ and $E_i[\Psi_i]$}. 
\begin{itemize}
\item Typically density matrix elements are good descriptors. This is because effective Hamiltonians are linear functions of them.
\item Basis functions and relaxation.
\item Energy-dependent terms can be included by deviating from the linear {\it ansatz}
\end{itemize}
 
\paragraph{Assess descriptors}. 
\begin{itemize}
\item Check distributions
\item Check correlations between descriptors. Large correlation $\rightarrow$ either not enough sampling or your model is overcomplete.
\item Check energies distribution
\item If you have multiple energies with the same descriptor $\rightarrow$ either take the lowest energy one or reconsider the descriptor space.
\end{itemize}

\paragraph{Ansatz: $E_i \simeq \sum_i c_i d_i$} 
\begin{itemize}
\item 
\end{itemize}

\paragraph{Fit bounding plane}
\begin{itemize}
\item Removing redundant descriptors: LASSO, or matching pursuit
\item The space of descriptors is smaller than the space of wave functions, so some wave functions will map to the same point in descriptor space. In that case, the low-energy space is given by the bounding plane.
\end{itemize}

