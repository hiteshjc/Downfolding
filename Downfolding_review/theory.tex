\documentclass[prl]{revtex4-1}

\usepackage{amsmath}
\usepackage{amsthm}
\usepackage{amssymb}
\usepackage{graphicx}
\usepackage{braket}
\usepackage{xcolor} 
\usepackage{tikz}
\newcommand{\HJC}[1]{{\color{red}{\bf HJC: #1}}}
\newcommand{\lucas}[1]{{\color{blue}{\bf LKW: #1}}}
\newtheorem{theorem}{Theorem}
\newtheorem{definition}{Definition}

\begin{document}

\section{Downfolding recast as a compression of the energy functional}

\subsection{Theory} 

Suppose we start with a quantum system with Hamiltonian $H$ and Hilbert space ${\cal H}$.

\begin{definition}
$E[\Psi] = \frac{\bra{\Psi}H\ket{\Psi}}{\braket{\Psi|\Psi}}$
\end{definition}


\begin{definition}
Let ${\cal H}$ be a Hilbert space. Then ${\cal LE}(H,N)$ is a subset of ${\cal H}$ spanned by $N$ vectors given by the lowest energy solutions to $H\ket{\Phi_i}=E_i{\Phi_i}$. 
\end{definition}

\begin{definition}
$H_{eff}$ is an operator on the Hilbert space ${\cal LE(H,N)}$.	\lucas{Does this need to be linear?} 
\end{definition}


\begin{definition}
The effective model $E_{eff}[\Psi]=\frac{\bra{\Psi}H_{eff}\ket{\Psi}}{\braket{\Psi|\Psi}}$ is a functional from ${\cal LE} \rightarrow \mathbb{R}$
\end{definition}



\begin{theorem}
\label{theorem:criticalpoint}
$E[\Psi]$ has a critical point only where $\Psi$ is an eigenstate of $H$.
\end{theorem}
\begin{proof}
This has been proven already by \lucas{find citation} 	
\end{proof}


\begin{theorem} 	
If $E_{eff}[\Psi] = E[\Psi], \forall \ket{\Psi} \in {\cal LE}$, then $H_{eff} \ket{\Phi_i} = E_i\ket{\Phi_i}$ for all eigenstates $\ket{\Phi_i}\in {\cal LE}$.
\end{theorem}
\begin{proof}
	Suppose that $\ket{\Phi_i}$ is an eigenstate of $H$. Then 
	\begin{equation}
	\left.\frac{\partial E[\Psi]}{\partial\Psi}\right|_{\Psi=\Phi_i} = 0 
	\end{equation}
\lucas{We have to project the derivative into the low-energy subspace. If the functional derivative of $E$ is zero at a point, then certainly the functional derivative of $E_{eff}$ is zero.  } 
Then
\begin{equation}
	\left.\frac{\partial E_{eff}[\Psi]}{\partial\Psi}\right|_{\Psi=\Phi_i} = 0 
\end{equation}
for $\ket{\Psi} \in {\cal LE}$. Using Theorem~\ref{theorem:criticalpoint}, $H_{eff} \ket{\Phi_i} = E_i\ket{\Phi_i}$. 
Similarly, if the derivative is not zero, then the wave function must not be an eigenstate.
\end{proof}

We have thus reduced the problem of finding an effective Hamiltonian $H_{eff}$ that reproduces the low-energy spectrum of $H$ to matching the corresponding energy functionals $E[\Psi]$ and $E_{eff}[\Psi]$. 
An important result of this is that it is not necessary to diagonalize either of the Hamiltonians; one must only be able to select wave functions from the low-energy space ${\cal LE}$.
As we shall see, this can be substantially easier than attaining full eigenstates.


The theory presented above maps coarse-graining into a functional approximation problem. 
This is still rather intimidating, since even supposing one can generate wave functions in the low-energy space, they are still complicated objects in a very large space.
An effective way to accomplish this is through the use of descriptors, $d_i[\Psi]$, which map from ${\cal H} \rightarrow \mathbb{R}$.
Then we can approximate the energy functional as follows
\begin{equation}
E_{eff}[\Psi] \simeq \sum_i f_i(d_i[\Psi]),
\end{equation}
where $f_i$ are some parameterized functions.
\lucas{This is a compression operation.} 

\subsection{An example} 

\lucas{Simple example: Hubbard $\rightarrow$ Heisenberg?} 

\end{document}
