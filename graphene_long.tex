\documentclass[aip,jcp,twocolumn,10pt]{revtex4-1}
\usepackage{dcolumn}
\usepackage{bm}
\usepackage{color}
\usepackage[english]{babel}
\usepackage{float}
\usepackage{amsmath}
\usepackage{graphicx}
\usepackage{times}   %% Times Roman font
\usepackage{esint}
\usepackage{subfigure}
\usepackage[unicode=true,pdfusetitle,
 bookmarks=false,colorlinks=true,citecolor=blue,urlcolor=blue,linkcolor=red]{hyperref}

\makeatletter

\@ifundefined{textcolor}{}
{%
 \definecolor{BLACK}{gray}{0}
 \definecolor{WHITE}{gray}{1}
 \definecolor{RED}{rgb}{1,0,0}
 \definecolor{GREEN}{rgb}{0,1,0}
 \definecolor{BLUE}{rgb}{0,0,1}
 \definecolor{CYAN}{cmyk}{1,0,0,0}
 \definecolor{MAGENTA}{cmyk}{0,1,0,0}
 \definecolor{YELLOW}{cmyk}{0,0,1,0}
}

\@ifundefined{date}{}{\date{}}
\AtBeginDocument{
  \def\labelitemi{\(\rhd\)}
}
\makeatother

\setlength{\belowcaptionskip}{-7pt}
\newcommand{\SAVE}[1]{}
\newcommand{\HJC}[1]{{\color{RED}{\bf HJC: #1}}}
\newcommand{\Ncal}{{\mathcal N}}
\newcommand{\T}{{\mathbf{T}}}
\newcommand{\Jbq}{{J_{bq}}}
\newcommand{\Jbl}{{J_{bl}}}

\begin{document}
\renewcommand\abstractname{}

\title{Low Energy Effective Hamiltonians from \emph{ab-initio} wavefunctions}
\author{Hitesh J. Changlani}
\affiliation{Department of Physics, University of Illinois at Urbana-Champaign, 1110 West Green St, Urbana IL 61801, USA}
\author{Huihuo Zheng}
\affiliation{Department of Physics, University of Illinois at Urbana-Champaign, 1110 West Green St, Urbana IL 61801, USA}
\author{Lucas K. Wagner}
\affiliation{Department of Physics, University of Illinois at Urbana-Champaign, 1110 West Green St, Urbana IL 61801, USA}
\date{\today}

\begin{abstract}
We propose a way of obtaining effective low energy Hubbard-like Hamiltonians from \emph{ab initio} 
Quantum Monte Carlo calculations for molecular and extended systems. 
The Hamiltonian parameters are fit to best match the \emph{ab-initio} 
two-body density matrices (computed in a localized basis) 
and energies of the ground and various excited states. 
For benzene (a finite system), we find 
good agreement with experimentally available energies 
without using any experimental inputs. 
For graphene, a two dimensional solid (extended system) 
with periodic boundary conditions, we find the effective onsite 
Hubbard $U^{*}/t$ to be $1.1 \pm 0.1$, comparable to 
a recent estimate based on the constrained-RPA approximation. For molecules, such 
parameterizations enable calculation of excited states that are usually not 
accessible within ground state approaches. For solids, the effective 
Hamiltonian enables large scale calculations using techniques 
designed for lattice models.  
\end{abstract}

\maketitle
\section{Introduction}
\label{sec:introduction}
Reliably simulating quantum systems remains a challenge 
for the physics and chemistry communities. Calculations on real materials 
are primarily carried out with density functional theory (DFT), 
which in principle is exact, but in practice is limited by the quality of the exchange functional. 
On the other hand are approximate model Hamiltonians (such as the Hubbard model~\cite{Hubbard}), 
which describe the low energy physics solely in terms of the valence electrons and 
are crucial to our understanding of physical phenomena such 
as antiferromagnetism and high temperature superconductivity. 
However, transitioning from actual materials to an effective Hamiltonian on a lattice relies 
on physical insight and/or fits to experimental data, and this notion is not always rigorously well justified.~\footnote{
While terms introduced in a lattice model are often very well motivated by the 
underlying physics and the symmetries of the problem, 
the magnitude (and sometimes even sign) the couplings is unknown.}
 
This effective reduction of the complexity of the calculation is known as "downfolding" 
and is commonly carried out using DFT-based approaches~\cite{Pavirini,Dasgupta}. 
These methods use the electronic band structure to generate a tight binding model, 
but are generally silent on the role of interactions (i.e. two and higher body terms). 
Incorrect estimation of the strength of the interactions 
in the case of systems with strong electron-electron correlations leads to 
outcomes in clear violations of experiments. For example, the local density 
approximation (LDA) predicts undoped LaCu$_2$O$_4$ 
to be metallic instead of insulating, requiring adhoc inputs for 
the downfolding procedure~\cite{Kent_Hubbard}. 
Other schemes for magnetic systems involve fitting DFT energies for 
various spin configurations to effective Heisenberg~\cite{Valenti_kagome} 
and bilinear-biquadratic models~\cite{Spaldin}.    

Once an effective Hamiltonian in the \emph{reduced} Hilbert space 
is obtained, it can be used to perform a lattice-model calculation 
on a \emph{much} bigger system. Many techniques such as the density matrix 
renormalization group (DMRG)~\cite{dmrg_white}, tensor networks~\cite{tps_nishino,Vidal_MERA,
TPS_review,Changlani_CPS,Neuscamman_CPS,mezzacapo,Marti}, 
dynamical field theory (DMFT), density matrix embedding (DMET)~\cite{DMET_2012,Chen_DMET} 
and lattice quantum Monte Carlo (QMC) methods~\cite{Sandvik_loops,Blankenbecler,Alavi_FCIQMC,SQMC} 
have been developed to deal with such model Hamiltonians. 
This multi-step modeling procedure is needed since the \emph{ab-initio} 
calculations for a given system size are, in general, much more 
computationally expensive than the equivalent lattice calculations. 
Large sizes are crucial to study finite size effects, and in turn 
theoretically establish the presence of a phase. For example, in the case of 
frustrated magnetic systems, many nearly-degenerate states exist some of which can be 
stabilized only on very large unit cells~\cite{Marston_Zeng}. 
In addition, excited states and dynamical correlation functions have traditionally 
been difficult in \emph{ab-initio} approaches, 
but have seen progress for lattice model methods~\cite{Daley_tDMRG, White_tDMRG}.

Various efforts have been taken to develop accurate downfolding 
methods. For example, Aryasetiawan et al.~\cite{Aryasetiawan} 
have laid out a post-DFT framework for obtaining 
static and frequency dependent Hamiltonians using the 
constrained-RPA approach. While the method has been applied to 
a wide range of systems~\cite{Imada1,Imada2,Arya1,Arya2,Wehling_graphene}, 
it is only recently~\cite{RPA_Troyer} that its accuracy is being rigorously checked. 
Other approaches include traditional Lowdin downfolding 
within a stochastic approach~\cite{Tenno} and 
canonical transformation theory~\cite{White_CT,Yanai_CT}, 
also being actively pursued as quantum chemistry 
methods~\footnote{A special case of canonical transformation theory is 
the Schrieffer-Wolf transformation~\cite{Schrieffer_Wolff} used in the 
BCS theory of superconductivity}.  

In this paper, we apply the \emph{ab-initio} 
Quantum Monte Carlo (QMC) approach~\cite{Ceperley_Alder,Foulkes_review}, 
one of the most accurate and highly scalable wavefunction based electronic-structure 
methods. The QMC method works directly in the continuum and 
explicitly introduces correlation into the Slater determinant obtained 
from DFT or Hartree Fock calculations. For energies of eigenstates, obtained 
by stochastic projection, the only systematic error is in the assumption of the 
nodes of the many-body wavefunction: in practice, 
this approximation is estimated to miss $ 5 - 10 \%$ 
of the correlation energy. Correlation functions and density matrices are generally 
less accurate, but good wavefunctions can bring down the errors significantly. 
For arbitrary wavefunctions which can be evaluated in polynomial time, 
the \emph{only} error is purely statistical and 
can be brought down to arbitrarily small values by increased sampling.

While all results presented here utilize data generated from 
\emph{ab-initio} QMC, the idea of fitting data from many-body wavefunctions to models applies 
to other methods as well. Information from many-body wavefunctions detects 
inadequacies of a proposed model: the deviations suggest the form of additional missing terms. 
Crucially, a wavefunction-based method does not differentiate between kinetic 
and potential energy terms: all terms are treated the same way. 
The approach is complementary to semi-empirical methods that 
fit models solely to available energy spectra. 

\begin{figure}[htpb]
\centering
\includegraphics[width=1\linewidth]{./effham.pdf}
\caption{Schematic for downfolding. The full Hamiltonian H is defined in the space of active (partially occupied), 
core (mostly occupied) and virtual (mostly unoccupied) orbitals. The objective is to map the physics 
of the original system to that of the effective one $\tilde{H}$ defined only in the active space.}
\label{fig:effham} 
\end{figure}	

In the rest of this paper, we develop and demonstrate our downfolding formalism. 
Since the concepts require an understanding 
of the \emph{ab-initio} calculations, we begin by discussing important details of the methods used.
Then, a set of criteria for a low-energy effective Hamiltonian are 
established and the fitting procedure is motivated and 
explained. We test our proposed ideas for the benzene molecule: 
a 30 electron problem is downfolded to a 6 electron one by describing the system with 
a single-band extended Hubbard model. The energy gaps of this model Hamiltonian 
are compared to experimentally available energies \emph{without} 
using any additional external inputs. 
As an example of application to solid materials, 
the question of the strength of Coulomb correlations in graphene is addressed. 
Using periodic unit cells, along with a consideration of finite size effects, 
we estimate that the effective onsite interaction is $U^{*}/t = 1.1 \pm 0.1 $, 
in the same range as a recently reported estimate~\cite{Schuler_graphene} 
based on the constrained-RPA method. Finally, we conclude by proposing 
future directions and potential applications of the method introduced. 

\section{Methods}
In the present section we discuss the various methods we used to generate our data. 
While most of our discussion is specific to QMC, 
the measures used can also be calculated in 
almost any other wavefunction-based quantum chemistry method. 
The reader interest in the main ideas may skip this section and directly proceed 
to the next one. 

All the \emph{ab-initio} QMC calculations were carried 
out with the QWalk package~\cite{QWALK}.
   
\subsection{Variational and Fixed Node Diffusion Quantum Monte Carlo}
\emph{Ab-initio} Quantum Monte Carlo comprises of a suite of methods that efficiently sample 
the phase space of $N$ electrons each moving in 3-dimensional real space. 
When the wavefunction as a function of $3N$ coordinates is known, the phase space
can be sampled with the variational Monte Carlo (VMC) using Metropolis algorithms.
For ground state calculations, the Diffusion Monte Carlo (DMC) method, 
based on imaginary time evolution of the Schroedinger equation, is formally exact but in practice severely 
limited by the numerical sign problem. This problem ceases to exist if one knew the exact location 
of the nodes (zeroes) of the many-body wavefunction. Thus, the optimal 
strategy for very accurate calculations is 
to generate a good trial wavefunction, and optimize its parameters to 
minimize its variational energy. Then use this wavefunction to "fix the nodes" (which may 
only approximately correspond to the exact nodes) and perform a DMC calculation under this constraint.  
This last variant is called the fixed node DMC (FN-DMC) method and is known to be 
very accurate for a large variety of systems. While some more details are discussed here, 
we refer the interested reader to Ref.~\cite{Foulkes_review} for an exhaustive 
review of concepts and applications.

A typical QMC calculation was carried out as follows. 
First, perform DFT calculations with the B3LYP~\cite{B3LYP} or 
PBE functionals~\cite{PBE} using the GAMESS software~\cite{GAMESS} 
for molecules or the CRYSTAL code~\cite{CRYSTAL} for solids.  
The lowest energy DFT orbitals provide the Slater determinant part of the trial 
wavefunction. For molecules, a multi-determinantal wavefunction is generated by performing 
a CI-Singles Doubles calculation from the reference Slater determinant. 
Once this is done, a Jastrow factor is introduced, resulting in the ansatz,
\begin{equation}
	\psi_T (r_1,r_2,....r_N) = \mathcal{J} \sum_{i} d_i D_i
\end{equation} 
When we desire eigenstates, the full wavefunction 
(i.e. parameters in the Jastrow $\mathcal{J}$ 
and the coefficients $d_i$ of the determinants $D_i$ 
are optimized to get the best possible variational energy within the ansatz chosen) 
using a technique introduced by Umrigar and coworkers~\cite{Umrigar_optimization} 
with an efficient implementation for calculating observables 
by Clark et al.~\cite{Clark_multidet}.

The variational energy is calculated via Metropolis sampling of $|\psi_T|^{2}$, 
\begin{equation} 
 E_{VMC} \equiv \frac{ \langle \psi_T | H | \psi_T \rangle} {\langle \psi_T | \psi_T \rangle} = \frac{ \int |\psi_T ({\bf{R}})|^2 \frac{H \psi_T({\bf{R}})}{\psi_T({\bf{R}})} \;\;\; d{\bf{R}} } { \int |\psi_T ({\bf{R}})|^2 d {\bf{R}}}
\end{equation}
where $\bf{R}$ is a compact notation for the 3-N 
coordinates $(r_1,r_2,....r_N)$ and $H \psi_T({\bf{R}})/\psi_T({\bf{R}})$ is the "local energy". 
With this trial wavefunction, we perform Diffusion Monte Carlo 
calculations under the fixed node constraint, (FN-DMC) 
to get the projected estimate of the energy, 
\begin{equation}
E_{DMC} \equiv \frac{ \langle \psi_T | H | \psi \rangle} {\langle \psi | \psi_T \rangle} = \frac{ \int \psi_T ({\bf{R}}) \psi({\bf{R}}) \frac{H \psi_T({\bf{R}})}{\psi_T({\bf{R}})} d {\bf{R}} }{ \int \psi_T ({\bf{R}}) \psi ({\bf{R}}) d {\bf{R}}}
\end{equation}
where $\psi \equiv \exp (-\beta H) \psi_T$ is obtained by a stochastic projection 
of $\psi_T$ under the constraint that $\psi$ and $\psi_T$ have the same sign everywhere. 
Thus, at the cost of a biased answer, one works with the positive definite 
distribution $\psi \psi_T$ without a Monte Carlo sign problem. (This bias 
is not a problem when eigenstates are not needed: the mixed distribution 
corresponds to some positive-definite sampling function.  We then view the 
"energy" as not an eigen-energy, but simply as an expectation value over this mixed distribution which can be 
calculated to arbitrary accuracy. This interpretation will be useful for the 
effective Hamiltonian construction discussed in a later section.)   

We now discuss measurements in the QMC methods. The two estimators that are computed for a generic operator $\hat{O}$ 
are the "pure" (VMC) and mixed (projected) estimators, 
\begin{equation}
         \langle \hat{O} \rangle_{VMC} \equiv \frac{ \langle \psi_T | O | \psi_T \rangle} {\langle \psi_T | \psi_T \rangle} 
\;\;\;\; \langle \hat{O} \rangle_{mix} \equiv \frac{ \langle \psi_T | O | \psi   \rangle} {\langle \psi_T | \psi\rangle}
\end{equation}
The projected estimator of an operator is equal to the pure estimator 
only in two limits: (1) when $\psi_T$ is the exact wavefunction or (2) when the 
operator commutes with the Hamiltonian. Thus, when dealing with eigenstates 
it has been shown by Ceperley that it is accurate to use the "extrapolated" 
estimator,
\begin{equation}
	\langle \hat{O} \rangle_{\text{extrap}}  = 2 \langle \hat{O} \rangle_{\text{mix}} - \langle \hat{O} \rangle_{\text{VMC}}
\end{equation}
For good wavefunctions, all these estimators must approach the same value: practically, 
however, when very high accuracy is needed, one may not be in this regime. 

For the purpose of construction of effective Hamiltonians, the operators measured are the two-body reduced density matrix (2-RDM) 
elements, 
\begin{widetext}
\begin{eqnarray}
	\rho_{ijkl} &\equiv& \langle c_{i}^{\dagger} c_{j}^{\dagger} c_l c_k \rangle = \sum_{ a \neq b} \int \phi^{*}_k (r'_a) \phi^{*}_l (r'_b) \phi_i(r_a) \phi_j(r_b) \Psi^{*}({\bf{R}''_{ab}})~\Psi ({\bf{R}}) dr'_a dr'_b d {\bf{R}}
\label{eq:2rdm}
\end{eqnarray}
\end{widetext}
where $ {\bf{R}''_{ab}} = (r_1,r_2, r'_a..., r'_b,....r_N)$ is the set of coordinates with 
the many-body wavefunction and $\phi_i(r)$ 
are pre-decided one-particle wavefunctions (orbitals) indexed by compact 
label $i (j,k,l)$ which accounts for space and spin. 
The mixed estimator equivalent of Eq.~\eqref{eq:2rdm} 
is obtained much like that for the energy. 
More details of this computation 
have been previously discussed elsewhere by one of us~\cite{Wagner_JCP}. The chosen set of orbitals 
is often the Wannier-like localized functions: this property 
makes it convenient to define Hubbard-like models. 
We explain their construction next. 

\subsection{Localized orbitals}
\label{subsec:wannier}
Localized orbitals often provide an intuitive way of understanding an electronic 
system in terms of electron hops and on-site or inter-site repulsions. 
Thus, many works have been devoted to this subject: ranging from the Linearized Muffin-Tin 
Orbital (LMTO) method~\cite{Andersen} to the maximally localized Wannier 
function construction~\cite{Marzari_Wf}. 
Orbital localization has also been widely discussed 
in the quantum chemistry literature. 

\begin{figure}[htpb]
\centering
\includegraphics[width=\linewidth]{./benzene_graphene.pdf}
\caption{(Left) One of the six symmetry equivalent $\pi$ orbitals in the 
benzene molecule obtained by localizing three bonding 
and three anti-bonding orbitals. The 1-RDM computed within QMC 
confirms that the $1 s$-like and the $2p - \sigma$ orbitals 
(not shown) are close to fully occupied. 
Hence only the half filled $\pi$ orbitals are used to fit a Hubbard-like Hamiltonian. 
(Right) Representative localized orbitals for the $4\times4$ unit cell of graphene.}
\label{fig:orbs} 
\end{figure}	

The idea is to first select a set of orbitals 
in a certain energy window: for solids bands close to the Fermi level, for 
molecules these are orbitals that are partially occupied. 
Then, a \emph{unitary} transformation is performed to optimize a 
pre-decided metric for localization.
In this work, we minimize the 
spread,
\begin{equation}
S = \sum_{n} \left( \langle \phi_n | r^2 | \phi_n \rangle - \langle \phi_n | {\bf{r}} | \phi_n \rangle^{2} \right)
\end{equation} 
where $\phi_n(r)$ are the desired localized orbitals related to the chosen set of 
orbitals $\Phi_i(r)$ by a unitary transformation: 
$\phi_n (r) = \sum_{i} U_{ni} \Phi_i(r)$  

For some systems, as we will see in the case of benzene and graphene, 
it is necessary to include unoccupied 
(virtual) orbitals to get well-localized orbitals of the right symmetry~\cite{Hansen}.
Thus the construction of localized orbitals is not a black-box procedure.
(For example, a variant could be to take an arbitrary subset of \emph{all} 
orbitals and demand localization of a smaller subset.)

\subsection{Lattice model calculations}
The lattice model calculations for Hubbard models of 
benzene and graphene at half-filling were carried out with a combination of our own codes 
and the freely available QUEST determinantal Quantum Monte Carlo package~\cite{QUEST}. 
For the honeycomb lattice half-filled Hubbard model, the determinantal QMC 
method is sign problem free and the results are \emph{exact} up to statistical errors.
A time step of 0.1 was chosen and $\beta$ (the imaginary time) was set to 20 for every calculation. 
The lattice-QMC calculations were performed with 5000 sweeps, with an additional 
2000 sweeps being used for equilibration.

\section{Criteria for a low energy effective Hamiltonian}
\label{sec:demands}
Our aim is to obtain a low energy effective Hamiltonian defined in 
the active space of electrons which is preferably (but not necessarily) 
described in terms of localized orbitals. In this basis, the criteria any 
effective low energy model Hamiltonian must satisfy are,
\begin{itemize}
\item (a) The reduced density matrices (RDM) of the ground and excited states (computed in a basis) 
          obtained from the \emph{ab-initio} calculation must match with that of the model calculation. 

\item (b) The energy spectra of the \emph{ab-initio} and model systems must match in the energy window of interest. 

\item(c) The model must be simple enough and must contain the essential physics to avoid overparameterization and overfitting. 
\end{itemize}

The concept of matching RDMs, criterion (a), 
has previously appeared in related contexts~\cite{Acioli,Zhou_Ceperley, Changlani_percolation} 
and in work by one of us~\cite{Wagner_JCP}. Most physical properties, 
such as the charge and spin structure factors, are functions of the 2-RDM.
Practically, it may be computationally expensive to get 
high-order RDMs: in this study we use the matching condition only on the 
2-RDM $\rho_{ijkl} \equiv \langle c_i^{\dagger} c_j^{\dagger} c_l c_k \rangle$ where $i,j,k,l$ are orbital indices 
(including space and spin)~\footnote{Note that good quantum numbers such as 
total spin and angular momentum of a pure state are functions of the \emph{full} 
2-RDM (if the Hamiltonian has those same symmetries) and not the \emph{partial} ones we use for matching.}.
This criterion automatically ensures that 
the combined number of electrons occupying the orbitals 
is equal to those in the model Hamiltonian. (If any 
input state does not satisfy the condition of 
expected electron number, then it can not be described by the effective Hamiltonian.)  

The importance of excited state energies 
used in the fitting, criterion (b), is easily highlighted by the fact that 
the wavefunctions (and hence their two-body density matrices) 
are invariant to many kinds of terms that enter the Hamiltonian.
For example, the transformation, 
\begin{equation}
	H' \rightarrow H + \alpha S^2 + \beta n + \gamma S^2 n 
\end{equation}
is, by construction, consistent with all the 2-RDM data 
for any $\alpha$, $\beta$, $\gamma$ 
for systems which have spin symmetry and conserve particle number.
Imposing certain "physical constraints" on the form of the 
interactions can eliminate the need for this criterion. 
To give a concrete example, consider wavefunction data 
generated from the ground state of an unfrustrated 
Heisenberg Hamiltonian in a bipartite lattice~\footnote{A bipartite 
lattice is one with two sublattices $A$ and $B$ with 
only $A-B$ connections but no $A-A$ or $B-B$ ones.}, 
$ H = J \sum_{\langle i,j \rangle}\bar{S}_i \cdot \bar{S_j}.$ 
where $\langle i,j \rangle$ refer to nearest neighbor pairs.
Then adding $\alpha S^2$ gives the same correlators (density matrices) 
of the ground state, as long as $\alpha$ is small enough to not cause 
energy crossings i.e. not make an original 
excited state the new ground state. This additional 
term has the effect of introducing long range Heisenberg couplings. 
Moreover, the effective Hamiltonian is not unique: 
the Lieb-Mattis model~\cite{LM} $H = S_A \dot S_B$ 
(where $A$ and $B$ refer to sublattice spins), is also 
known to reproduce the low-energy limit of the Heisenberg model. 
Thus, imposing the requirement that the Hamiltonian has the 
nearest-neighbor form constrains $\alpha$ to zero and picks 
one particular model. Similar arguments should 
apply to long-range Hubbard models in homogenous 
systems where a physical constraint is that density-density 
interaction must decrease monontonically with distance between orbitals. 

While having many excited states from \emph{ab-initio} calculations 
is desirable, we note that certain methods only work best for 
determining ground states in symmetry sectors. 
In the case of FN-DMC, 
some other excited states can also be achieved by fixing their 
nodal structure (which is different from that of the ground state). 
Our practical experience suggests that attempts to obtain 
eigenstates were often associated with broken symmetries. 
This happens for a variety of reasons: choice of DFT orbitals, 
contamination of the spin-structure by the 
Jastrow and/or incomplete optimization. 
These limitations suggest that it is 
important to reduce the dependence of the desired 
method only on eigenstates. Instead, our philosophy is to 
use the \emph{available} information from arbitrary 
low-energy states to our advantage. 
(Moreover, if \emph{all} eigenstates were known, the utility of 
generating an effective Hamiltonian might be diluted anyway.)
  
Given a finite amount of data, we note that criteria (a) and (b) 
do not always uniquely determine an effective Hamiltonian, 
even within the ansatz chosen. For example, one may demand that 
the cross expectation values of certain operators 
(say $\langle \psi_1 | c_i^{\dagger} c_j | \psi_2 \rangle$) 
between the \emph{ab-initio} and model calculations must match. 
This information could also be added as an 
additional criterion for matching. 

Finally, it is important to build confidence 
in this Hamiltonian by comparing its outputs with respect 
to data not used in the fitting. 
To partly remedy this, we propose the 
heuristic criterion (c). As is the case with many situations 
involving fititng, we desire 
the minimal model that explains the data to 
some reasonable accuracy. This has the effect of making such a 
model Hamiltonian useful in other situations i.e. transferable.

\section{Hubbard interactions as fitting parameters}
\label{sec:fitting}
What is the meaning of the Hubbard $U$ (and other interactions)?
Is there a fundamental definition or are they fit parameters to a chosen model? 
We take the latter viewpoint and thinking this way helps reconcile 
widely varying estimates that exist in the literature. 

%A common expression for the generalized-two body interaction is taken to be, 
%\begin{equation}
% V_{ijkl} = \int  \phi_i (r) \phi_j (r')  W (r,r') \phi_k (r) \phi_l (r')
%\end{equation}
%where the modified interaction $W(r,r')$ is obtained from a model for screening 
%relating it to the bare Coulomb interaction $V(r,r')$. 
The Hubbard $U$ depends on the energy range 
over which one is interested in estimating it. It is an 
"effective interaction" that encodes the effect of many others 
which would otherwise be needed in a completely static model.
Even in the low frequency static limit, estimates 
of the Hubbard $U$ can vary over large factors~\cite{Fei_Lin} 
(of the order of $2$ or more): this is partly attributed to insensitivity 
of correlation-functions in the ground state to its precise value. 
Any inaccuracy in the \emph{ab-initio} calculation reflects as 
large changes in the estimated $U$. This point 
will become clearer in later sections of the paper.
 
In the fitting approach, the idea is to use a sufficiently 
large data set such that the proposed Hamiltonian has "learnt" most 
relevant features of the low-energy physics, after which the parameters 
are used for understanding other properties. 
This reconstruction depends on the kind of data input to the method 
and the presence of linear dependencies. 
We present a formulation which puts the statements above 
on a more rigorous mathematical footing.

\subsection{Fitting procedure: $Ax=E$ method}
\label{sec:AxE}
Consider a set of \emph{ab-initio} 
energies $E_s$ (in general, expectation values of the Hamiltonian) 
and corresponding 1 and 2 body RDMs $\langle c_i^{\dagger} c_j \rangle_s$ 
$\langle c_i^{\dagger}c_j^{\dagger} c_l c_k \rangle_s$ 
for various \emph{arbitrary} low-energy states characterized by index $s$. 
Assume a model 2-body Hamiltonian parameterized by (yet unknown) 
renormalized parameters $t_{ij}$ (one body part) 
and $V_{i,j,k,l}$ (2-body part) along with a constant term $C$, 
the total number of parameters being $N_p$. Then for each state $s$, 
we have the equation, 
\begin{equation}
	E_s = \langle H \rangle_s = C + \sum_{ij} t_{ij} \langle c_i^{\dagger} c_j \rangle + \sum_{ijkl} V_{ijkl} \langle c_i^{\dagger}c_j^{\dagger} c_l c_k \rangle  
\end{equation}
where we have made the assumption that the chosen set of orbitals 
is capable of explaining all energy differences. The constant $C$ is from energetic contributions 
of all other orbitals which are (almost) decoupled from the chosen set.  
 
We then perform calculations for 
$M$ (possibly random) low-energy states which are not necessarily eigenstates.  
The objective is to explore various parts 
of the low-energy Hilbert space which show variations in the 2-RDM elements. 
Since the same parameters describe all $M$ states, they must 
satisfy the linear set of equations, 
\begin{eqnarray}
\left(
\begin{array}{c}
E_1 \\
E_2 \\
E_3 \\
... \\
... \\
... \\
... \\
E_M
\end{array}
\right) =
\left(
\begin{array}{ccccc}
1 & \langle c_i^{\dagger}c_j \rangle_{1}  & .. & \langle c_i^{\dagger}c_j^{\dagger}c_l c_k \rangle_{1} & .. \\
1 & \langle c_i^{\dagger}c_j \rangle_{2}  & .. & \langle c_i^{\dagger}c_j^{\dagger}c_l c_k \rangle_{2} & .. \\
1 & \langle c_i^{\dagger}c_j \rangle_{3}  & .. & \langle c_i^{\dagger}c_j^{\dagger}c_l c_k \rangle_{3} & .. \\
1 & \langle c_i^{\dagger}c_j \rangle_{4}  & .. & \langle c_i^{\dagger}c_j^{\dagger}c_l c_k \rangle_{4} & .. \\
1 & ....                                  & .. & ..                                                    & .. \\
1 & ....                                  & .. & ..                                                    & .. \\
1 & \langle c_i^{\dagger}c_j \rangle_{M}  & .. & \langle c_i^{\dagger}c_j^{\dagger}c_l c_k \rangle_{M} & .. \\
\end{array}
\right) \left(
\begin{array}{c}
C           \\
t_{ij}      \\
..          \\
V_{ijkl}    \\
..
\end{array}
\right)
\end{eqnarray}
which can be written more compactly as,
\begin{equation}
	{\bf{E}} = A {\bf{x}}
\end{equation}
where $ {\bf{E}} \equiv (E_1,E_2,...E_M)^{T}$ 
is the $M$ dimensional vector of energies, $A$ is the $M \times N_p$ matrix composed 
of density matrix elements and $ {\bf{x}} \equiv (C,t_{ij}....V_{ijkl}...)^T$ 
is a $N_p$ dimensional vector of parameters.
This problem is overdetermined for $M>N_p$, which is the regime we expect to work in.

\begin{figure}[htpb]
\centering
\includegraphics[width=1\linewidth]{./hamfit.pdf}
\caption{Methods employed for downfolding involving Hamiltonian parameter fitting (a) Matching 
the reduced density matrices (RDMs) of energy-eigenstates 
(b) The "\emph{Ax=E}" method uses RDMs and energies of arbitrary states. }
\label{fig:hamfit} 
\end{figure}	


In the case of any imperfection in the model (which is the most common case), 
the equality will not hold exactly 
and one must then instead minimize the norm, $\mathcal{N}$,
\begin{equation}
	\mathcal{N} \equiv ||A\bf{x}-\bf{E}||^2
\label{eq:norm}
\end{equation}
The nice property of this cost function is that it can be minimized
in a single step by using the method of least squares, employing the singular 
value decomposition of matrix $A$, which also encodes exact (or near-exact) linear dependences. 
Thus, the quality of the fit can be directly judged 
by assessing (1) the singular values of the $A$ matrix and (2) 
the value of the cost function itself i.e. the deviations of the input and fitted energies.
We will refer to this method as the "Ax=E" method at various points in the paper.

The cost function~\eqref{eq:norm} is unlike that used when matching 
canonical density operators,
\begin{equation} 
\frac{1} {Z^{m}} \exp(-\beta H^{m})  = \frac{1} {Z^{a}} \exp(-\beta H^{a})
\label{eq:Z_Zstar}
\end {equation}
where indices $m$ and $a$ refer to the model and ab-initio (true) systems, $\beta$ 
is an inverse temperature and $Z$ is the partition function
In contrast, our "energy-weighting" function is a box-function: it weighs all the energies 
in a given window equally. The form of the cost function (including the 
weighting function) is unimportant in the limit of a perfect model.

In summary, our approach solves 
linear equations in terms of the unknown parameters, by using the density matrices 
from an \emph{ab-initio} method. This concept works only when we 
know the model Hamiltonian that describes the data, and all we want is the numerical 
values of the parameters. Thus to verify the validity of the calculation, 
one must use the derived parameters and calculate many-body eigenstates 
in the model and check their correspondence with the \emph{ab-initio} 
calculations.

The matrix $A$ gives a very natural basis to understand "renormalization" effects.
For example, consider a set of wavefunctions, that show that the correlator 
$\langle n_0 n_3 \rangle$ does not change significantly. This would lead to the 
corresponding column of matrix $A$ being identical (up to a scale factor) 
to the first column of 1's. Physically, this would correspond to the coupling constant 
$V_{03}$ being unimportant (it can take any value including 0) and its 
effect can be completely absorbed into the constant shift term. 
(This could also alternatively mean that the input 
data is correlated and does not provide enough information about $V_{03}$. 
In the case of non-zero "small variations", the value of this parameter 
will be quite sensitive to the data set and its quality.) 
Said differently, it is the \emph{variation}  
in an RDM element (across states in the energy window) 
that are important: \emph{not} their \emph{absolute} values.

The utility of the wavefunction approach is apparent in this 
formulation: the expectation values entering matrix $A$ 
can be calculated for arbitrary wavefunctions by Monte Carlo sampling.
We can also simply use variational Monte Carlo and corresponding energies
for constructing the $A$ matrix, however we may not get the desired accuracy. 

It is thus preferable to use the accurate FN-DMC method. In this case, one needs to modify the linear equations 
to use projected estimators. More specifically, we have, $E= \langle {\psi_T}^{s}| H | {\psi}^s \rangle$ 
and thus must use the projected estimates of the density matrix elements i.e. 
$\langle {\psi_T}^s | c_i^{\dagger} c_j | {\psi}^s \rangle$ and $\langle {\psi_T}^s| c_i^{\dagger}c_j^{\dagger} c_l c_k | {\psi}^s \rangle$ 
in the construction of the $A$ matrix. (The implicit normalization of these 
mixed estimates by $\langle {\psi_T}^s | \psi^{s} \rangle$ is assumed.) 
This projector formulation is also very amenable to 
coupled-cluster calculations which also work with projected energies 
and density matrices.

There is no need to worry about the bias in FN-DMC in our formulation. 
This is because we regard $\psi \psi_T$ as some \emph{arbitrary} positive sampling function 
associated with a low energy state and use the \emph{same} distribution for the 
evaluation of the density matrix elements. We are not concerned with whether 
an exact eigenstate is being sampled or not: we just desire a linear equation 
(for the unknown parameters) from this exercise. Thus the relationship between the FN-DMC energy 
and the projected density matrix elements is always an \emph{exact} relationship 
(up to errors coming in from statistics and from the 
assumption of the form of the Hamiltonian). 

\subsection{Simple application: Hubbard to Heisenberg model}
To demonstrate our formalism for a simple example, we 
consider the two site Hubbard model and fit information from the lowest two states 
to a Heisenberg model. (This is not technically downfolding, as there is no reduction 
of electron number). 

We analytically solve for all four eigenstates of the Hamiltonian,
\begin{equation}
H = -t \sum_{ij} c_{i,\sigma}^{\dagger} c_{j,\sigma} + \text{h.c.} + U \sum_{i=1,2} n_{i,\uparrow} n_{i,\downarrow}+ V n_1 n_2
\label{eq:two_hubbard}
\end{equation}
for two opposite spin electrons on two sites, where $t$ is the hopping, $U$ is the usual Hubbard term and $V$ is an inter-orbital density-density 
interaction. 
 
First notice that the triplet state $| \psi_t \rangle \equiv \frac{|\uparrow \;\; \downarrow \rangle - |\downarrow \;\; \uparrow \rangle }{\sqrt{2}} $ with energy $E_t=V$ and the state $|\psi_d \rangle \equiv \frac{|\uparrow \;\; \downarrow \rangle - |\downarrow \;\; \uparrow \rangle }{\sqrt{2}}$ with energy $E_d=U$, are exact eigenstates of the problem independent of the values of $t$ and $U$ and $V$. 

To get the other two states, write the Hamiltonian in the 
basis of $ |\psi_s \rangle \equiv \frac{1}{\sqrt{2}} \left( |\uparrow \;\; \downarrow \rangle + |\downarrow \;\; \uparrow \rangle \right) $ 
and $ | {\psi_d}^{'} \rangle \equiv \frac{1}{\sqrt{2}} \left( |\uparrow\downarrow 0 \rangle + |0 \uparrow\downarrow \rangle \right) $, 
\begin{eqnarray}
H =
\left(
\begin{array}{cc}
V   & -2t \\
-2t & U    \\
\end{array}
\right)
\end{eqnarray}
Then diagonalizing it (i.e. solving the quadratic equation), we get the energies to be, 
\begin{equation}
E_{\pm} = \frac{U + V \pm \sqrt{(U-V)^2+16 t^2} }{2}
\end{equation}

Consider the case of $V=0$, and $U>t>0$, the usual Hubbard model. 
The lowest energy corresponds to the singlet, $E_{-}$ and the 
corresponding eigenvector is 
\begin{equation}
	| \psi_{-} \rangle = \frac{2t}{\sqrt{4t^2+{E_{-}}^2}} |\psi_s \rangle  - \frac{E_{-}}{\sqrt{4t^2+{E_{-}}^2}} |{\psi_d}^{'} \rangle
\end{equation}
with the next excited state being the triplet $|\psi_t \rangle$.

We choose the Heisenberg form to fit to
\begin{equation}
	\tilde{H} = C + J S_1 \cdot S_2
\end{equation}
To determine the parameters $C$ and $J$, form the $2 \times 2$ $A$ matrix 
with the lowest two energy states,
\begin{equation}
\left(
\begin{array}{c}
 E_{-} \\ 
 E_t \\
\end{array}
\right)
=
\left(
\begin{array}{cc}
1   &  \frac{1} {4+(E_{-}/t)^2}	\\
1   & -3/4                      \\
\end{array}
\right)
\left(
\begin{array}{c}
 C \\
 J \\
\end{array}
\right)
\end{equation}
Using derived values of $E_s$ and $E_t=0$, we get,
\begin{equation}
	J = \frac{E_s (4+(E_{-}/t)^2)}{4+\frac{3}{4}(E_{-}/t)^2}
\end{equation} 
which to lowest order in $t/U$ is $J = -4t^2/U$.

Observe that the correlator for $\langle S_i \cdot S_j \rangle$ is \emph{not exactly} 1/4 
but only \emph{approximately} so. This is expected since the fluctuations from the 
high-energy states are not exactly zero: if it were, then it is equivalent to 
\emph{exactly} block -diagonalizing the Hamiltonian.

In terms of canonical transformations (here, equivalent to second order perturbation theory), 
the matrix element of the effective Hamiltonian between the singly occupied states is, 
\begin{eqnarray}
	\langle \uparrow\downarrow | \tilde{H} | \downarrow \uparrow \rangle &=& \sum_{n} \frac{ \langle \uparrow\downarrow|H| n\rangle\langle n |H| \downarrow \uparrow \rangle } {0-E_n} \nonumber \\ 
	                                                                     &=& \frac{-2 t^{2}}{U}
\end{eqnarray}
Since this matrix element must equal $J/2$ in the Heisenberg model, we arrive at the same result: 
namely $J = -4t^{2}/U $, as expected.
(To map a fermion model to a spin model, a Jordan-Wigner phase is often introduced 
which is fixed as there are no charge fluctuations. In this notation the singlet is 
$ \frac{1}{\sqrt{2}} \left( |\uparrow \;\; \downarrow \rangle - |\downarrow \;\; \uparrow \rangle \right) $, 
the triplet is $ \frac{1}{\sqrt{2}} \left( |\uparrow \;\; \downarrow \rangle + |\downarrow \;\; \uparrow \rangle \right) $ 
and $J=+4t^2/U$)

\subsection{Long range vs on-site Hubbard models}
Since the Hubbard $U$ is a fit parameter, its value 
can be vastly different depending on whether other long range 
interactions $V_{ij} n_i n_j$ are added or not. However, it is still 
meaningful to ask how the Hubbard $U$ in a long-range model 
and the effective $U^{*}$ in an on-site model are (approximately) 
related. 

Once again, this is most simply understood 
for the previous example of the two-site extended 
Hubbard model, where the $V$ term 
simply counters $U$: we have, 
\begin{equation}
	U^{*} = U - V
\end{equation}
A more generalized condition, for arbitrary range interactions, 
was derived~\cite{Schuler_graphene} by demanding that the canonical 
density operator of the short range model match the exact density operator 
as closely as possible, Eq.~\eqref{eq:Z_Zstar}. 
%Then a Pierls-Feynman-Bogoliubov 
%functional was defined and minimzed.

\section{Benzene molecule}
We show the workings of the described methods for the benzene molecule. 
Our choice of system is motivated by the simplicity of the system (one band model and presence of many symmetries) 
and the availability of experimental energies to compare to. 

% Fitting method
% Eigenstate
% Ax=b 
% for both ab initio and model hard to get eigenstates
% demonstration example of fitting on-site 
% to benzene
% Ax=b doesnt need eigenstates
% Here is how we use it for the same problem

DFT calculations were first performed in the TZVP basis with Burkatzki-Filippi-Dolg pseudopotentials~\cite{BFD}, 
using the molecular geometry that corresponds to a B3LYP optimized calculation. 
These serve as a starting point for the QMC calculations, discussed in the Methods section.
In the charge neutral sector, there are a total of 30 electrons (15 $\uparrow$  and 15 $\downarrow$ 
for the spin-singlet state) and our objective is to downfold 
this system to an effective one with 6 electrons (3$\uparrow$, 3 $\downarrow$). 

The model Hamiltonian is defined in the space of six $\pi$ orbitals (a representative has been shown in Fig.~\ref{fig:orbs}): 
these orbitals are not the atomic $p_z$ ones, but rather their renormalized versions.
They were obtained by localizing the highest three occupied B3LYP orbitals 
(from the $S=0$ ground state DFT calculation) and the lowest three virtual (unoccupied) orbitals with $\pi$ orbital symmetry: 
a well established procedure in the literature~\cite{Hansen}.  The overall phase (i.e. plus or minus sign) 
of these orbitals is adjusted to enable use of parameter symmetries directly when fitting.

\begin{table*}[htpb]
\begin{center}
\begin{tabular}{|c|c|c|c|c|c|c|c|c|}
\hline
Spin & Charge & DFT      &   SJ-VMC      &  SJ-DMC	 &  CISDJ-VMC    &  CISDJ-DMC  & $N_\uparrow,N_\downarrow$ & Used in Fit?\tabularnewline 
\hline							      
 0   & -2   & -37.4249   & -37.3899(6)   & x      	 & x             & -37.6221(8) &  x         & No \tabularnewline
\hline							      
 1/2 & -1   & -37.5891   & -37.5758(6)   & x      	 & -37.5944(6)   & -37.7066(6) &  x         & No \tabularnewline
\hline							      
 0   &  0   & -37.6303   & -37.6229(6)   & -37.7213(9)	 & -37.6352(6)   & -37.7259(9) & 2.96,2.96  & Yes \tabularnewline
\hline							      
 1   &  0   & -37.4634   & -37.4546(6)   & -37.5555(7)	 & -37.4814(6)   & -37.5707(7) & 3.94,1.98  & Yes \tabularnewline
     &      &            &               &          	 & -37.4561(6)   & -37.5479(6) & 3.94,1.98  & Yes \tabularnewline
     &      &            &               &       	 & -37.4531(6)   & -37.5470(6) & 3.94,1.98  & Yes \tabularnewline
\hline							      
 2   &  0   & -37.3203   & -37.2987(6)   & -37.3974(7)	 & -37.3141(6)   & -37.4020(7) & 4.92,1.00  & Yes \tabularnewline
\hline							      
 3   &  0   & -37.0378   & -37.0116(4)   & -37.1074(7)	 & -37.0118(4)   & -37.1083(7) & 4.88,0.02  & No \tabularnewline
\hline							      
 1/2 & +1   & -37.3881   & -37.2798(6)   & -37.3764(7)	 & -37.2993(5)   & -37.3832(6) & 2.95,1.98  & Yes \tabularnewline
\hline							      
 3/2 & +1   & -37.2329   & -37.1188(5)   & -37.2142(6)	 & -37.1350(6)   & -37.2203(6) & 3.92,1.00  & Yes \tabularnewline
\hline							      
 0   & +2   & -36.6949   & -36.6751(5)   & -36.7770(14)	 & -36.7095(6)   & -36.7930(15)& 1.98,1.98  & Yes \tabularnewline
\hline							      
 1   & +2   & -36.7359   & -36.7220(5)   & -36.8131(6)	 & -36.7404(5)   & -36.8209(6) & 2.94,1.00  & Yes \tabularnewline
\hline							      
 2   & +2   & -36.5048   & -36.4091(2)   & x             & x             & x           &            & Yes \tabularnewline
\hline 
\end{tabular}
\caption{Energy of different charge and spin eigenstates from DFT and QMC methods. 
SJ refers to the Slater Jastrow wavefunctions and CISD. All DMC calculations used a time step of 0.01 Ha$^{-1}$. 
The states that had significantly different occupations from 
those expected were not used in the fitting as they can not be 
described by an effective Hamiltonian.}
\end{center}
\label{tab:benzene}
\end{table*}

Table~\ref{tab:benzene} shows the QMC data for various (near) eigenstates of benzene. 
Most of these states along with many other non-eigenstates (not shown) 
constitute our data set for fitting a model. The calculations confirm 
the general expectation that significant energy gains are obtained by 
improving wavefunctions going from the single-Slater-Jastrow form to the 
multi-determinantal-Jastrow form (the determinants being selected 
from a CISD calculation). Moreover, the DMC calculations 
improve total energies significantly: typically, 
these values are $2-3$ eV below the corresponding VMC value.

The total electron count from the one-body density matrix is 
assessed to verify the validity of fitting to a six-$\pi$ orbital Hamiltonian. 
Table~\ref{tab:benzene} shows that for charge-neutral benzene, 
the singlet state ($S=0$) has up and down electron 
counts of 2.96 each, which are close to the expected values of 
3,3. For the $S=2$ state, roughly $9$ eV above the ground state, 
the deviations were slightly larger: the summed 
occupation numbers were 4.92 and 1.0 in comparison to the 
expected values of 5 and 1. However, for the $S=3$ state, these numbers deviate signficantly 
from the corresponding value in the model: 
almost by one integer. This indicates that the $S=3$ state is 
inadequately described by the proposed Hamiltonian and hence 
can not be used in the fitting procedure. 
This deviation is not completely unexpected: this state being 
$\sim 17$ eV above the ground state is a potentially "high-energy state". 
Said differently, the active space at this energy scale 
is radically different from that \emph{assumed} for the ground state and its 
\emph{low-energy} excitations.
Thus, our QMC data suggests that it is only reasonable for the effective 
Hamiltonian concept to hold only in an energy window of the order of $10 $ eV above the ground state 
(in the same charge sector).

\subsection {On-site Hubbard model}

We consider the Hubbard model for six orbitals of benzene, 
(idential to Eq.~\eqref{eq:two_hubbard}, with $V=0$ and generalized 
to six orbitals) where the notation $t$ will be used 
for the nearest-neighbor hopping and $U^{*}$ is the effective 
on-site Coulomb repulsion. 
We will discuss multiple ways of using reduced density matrix elements

\subsubsection{Hubbard $U^{*}/t$ from the 2-RDM}
An estimate of $U^{*}/t$ is obtained 
by directly matching the 2-RDM elements corresponding 
to the "double occupancy" correlator ($\langle n_{\uparrow} n_{\downarrow} \rangle$) 
of the \emph{ab-initio} and lattice-model calculations. This element equals $0.25$ 
for the non interacting case ($U^{*}=0$) and its value reduces for $U^{*}>0$.

\begin{figure}[htpb]
\centering
\includegraphics[width=1\linewidth]{./double_occupancy_benzene.pdf}
\caption{Double occupancy of the $\pi$ orbital of benzene 
as a function of $U^{*}/t$, computed in the half filled ground state. 
Comparisons are made with the values from the VMC and extrapolated estimators (see text) 
obtained from the ab-initio QMC calculation with Slater-Jastrow and CISD-Jastrow wavefunctions.}
\label{fig:rdm_benzene} 
\end{figure}	

\begin{figure*}[htpb]
\centering
\includegraphics[width=0.34\linewidth]{./rdm_benzene_hubbard.pdf}
\includegraphics[width=0.32\linewidth]{./nn_corr_benzene_hubbard.pdf}
\includegraphics[width=0.32\linewidth]{./ss_corr_benzene_hubbard.pdf}
\includegraphics[width=0.32\linewidth]{./err_rdm_benzene_hubbard.pdf}
\includegraphics[width=0.32\linewidth]{./err_nn_corr_benzene_hubbard.pdf}
\includegraphics[width=0.32\linewidth]{./err_ss_corr_benzene_hubbard.pdf}
\caption{Comparison of ab-initio Quantum Monte Carlo 
and Hubbard model correlation functions for 
the half filled singlet ground state for various values of $U^{*}/t$. 
The main panels show the  
(a) one body density matrix (b) density-density correlators 
(c) spin-spin correlators as a function of distance with respect to a
reference site (d)-(f) show the errors of the corresponding model 
correlation functions with respect to the ab-initio data. Most correlators are captured 
adequately, except for the nearest neighbor density-density correlator.} 
%This is taken as evidence for the need to go to longer range Hubbard interactions.}
\label{fig:HubbardU_benzene}
\end{figure*}	

Fig.~\ref{fig:rdm_benzene} shows the dependence of 
$\langle n_{\uparrow} n_{\downarrow} \rangle$ (computed in the ground state at half filling) 
on $U^{*}/t$ in the Hubbard model. The plot also indicates
the value of this correlator computed from various wavefunctions and estimators. 
The trends are consistent with our expectations: 
the Slater-Jastrow wavefunction (at the VMC level) underestimates 
the strength of the effective interactions, which is partly remedied by 
the extrapolated estimator (from DMC). However, the bias (systematic error) 
arising from the inaccuracy from the wavefunction is still large. 
This bias is reduced by the biggest multideterminantal-Jastrow 
(CISD-J) wavefunction we employed: the difference between the 
variational and extrapolated estimator is reduced (10-20\%). 

The value of $U^{*}/t$ is found to be extremely sensitive to the precise value of 
the double occupancy: a change of a few percent (i.e. from $0.23$ to $0.20$) 
changes our estimate from $0.5$ to $1.4$ (i.e. a factor of almost $3$). In general, 
this observation suggests that it is crucial to look at various other 
elements of the 2-RDM and to look at alternate ways 
of estimating Hubbard parameters.

\begin{figure}[htpb]
\centering
\subfigure[]{\includegraphics[width=\linewidth]{./Hubbard_5.pdf}}
\subfigure[]{\includegraphics[width=\linewidth]{./Hubbard_12.pdf}}
\caption{Fitted vs DMC energies for the on-site 
Hubbard model in energy windows of width 
(a) $5$ eV (b) $12$ eV above the ground state.}
\label{fig:Ax_b_Hubbard}
\end{figure}	

Fig.~\ref{fig:HubbardU_benzene} shows comparison of the QMC 
nearest-neigbor one body density matrix element 
and the density-density and spin-spin correlators and those for the Hubbard model 
for various values of $U^{*}/t$. The value of $U^{*}/t = 1.4$ gives the smallest errors 
for most observables, except for the nearest-neighbor
density-density correlator which prefers values of $U^{*}/t \sim 0$. 
In the limit that the model is perfect, all estimates must 
yield the same value: the differences reflect an inadequacy 
of the on-site Hubbard model in describing all the data. 
We take this as evidence for the need for long-range interactions.

\subsubsection{Hubbard $U^{*}/t$ from the Ax=E method}
At this point, we note that the idea of "matching density matrix elements"
is useful only for comparing \emph{exact} eigenstates. However, it is 
difficult to construct these with very high accuracy in the \emph{ab-initio}
calculations (and at times also for the equivalent model for large system sizes). 
This is why we appeal to the "Ax=E" method (introduced 
and explained in Sec.~\ref{sec:AxE}) 
which is relatively insensitive to the nature of the states input 
to the method: in general, \emph{any} low-energy state 
(within some energy range of the ground state) 
yields information about the effective Hamiltonian 
parameters.

\begin{figure*}[htpb]
\centering
\subfigure[]{\includegraphics[width=0.32\linewidth]{./t_var.pdf}}
\subfigure[]{\includegraphics[width=0.32\linewidth]{./t02_var.pdf}}
\subfigure[]{\includegraphics[width=0.32\linewidth]{./t03_var.pdf}}
\subfigure[]{\includegraphics[width=0.32\linewidth]{./U_var.pdf}}
\subfigure[]{\includegraphics[width=0.32\linewidth]{./V01_var.pdf}}
\subfigure[]{\includegraphics[width=0.32\linewidth]{./V02_var.pdf}}
\caption{Variation of RDM elements (indicated by "Var") for various states 
after subtracting out their averaged value. These elements couple to the relevant parameters 
in the lattice model. As is discussed in the text, we big variations are needed 
for the parameters to be estimated, else they can be taken to be effectively zero 
(as is observed to be the case for $t_{02}$).}
\label{fig:variations}
\end{figure*}	

For charge-neutral benzene, we construct the $A$ matrix 
by taking various states in $5$ and $12$ eV energy windows above the ground state. 
As is shown in Fig.~\ref{fig:Ax_b_Hubbard}, the $5$ eV window fit 
gives $t=2.94$ eV and $U^{*}=4.30$ ($U^{*}/t$ = 1.46) with a maximum error of 
0.2 eV and the $12$ eV fit gave $t=3.04$ eV, $U^{*}=4.54$ eV ($U^{*}/t = 1.50$) 
a maximum error of 0.5 eV. 
Studying different energy windows also provides a way of monitoring 
the "frequency dependence" of the Hubbard $U^{*}/t$. 

These ratios of $U^{*}/t$ are broadly consistent 
with that obtained from the double occupancy correlator. However, the 
values of $t$ is different from the fitted value of $t=2.54 $ eV reported by 
Bursill et al~\cite{Bursill}: the discrepancy could be attributed to 
(1) errors in experimental energies used in fitting 
(2) use of a long-range model for the interactions 
and (3) assumption that $t$ is not renormalized 
on going from a long range to short range model. 
Thus, we explore the long-range model in the next section 
as a way of obtaining a more refined model.

\begin{figure*}[htpb]
\centering
\subfigure[]{\includegraphics[width=0.31\linewidth]{./fit_VMC.pdf}}
\subfigure[]{\includegraphics[width=0.31\linewidth]{./fit_DMC.pdf}}
\subfigure[]{\includegraphics[width=0.33\linewidth]{./energies_benzene.pdf}}
\caption{Comparison of energies matched 
for states (not necessarily eigenstates) used in the fitting procedure 
involving their reduced density matrix elements (described in the text). 
The panels show (a) Fitted vs VMC energies (b) Fitted vs DMC energies (c) 
Energies of various eigenstates appearing for various model Hamiltonians 
vs experimental energy gaps.}
\label{fig:expt} 
\end{figure*}	

\begin{table*}[htpb]
\label{tab:param}
\begin{center}
\begin{tabular}{|c|c|c|c|c|c|}
\hline
Parameter  &  Bursill           &  VMC (This work) &  DMC (This work)  & $U^{*}$    &  $U^{*}$ (This work) \tabularnewline
\hline						   
\hline						   
% $E_{00}$  &    -               &   -		   &   -                &  -          &  -       \tabularnewline
%\hline						   
 $t$        &   +2.54            &   +3.05	   &   +2.86            &  +2.54      &  +2.94   \tabularnewline
%\hline						   
% $T_{02}$  &    -               &   0.0	   &   0.0              &  -          &    -    \tabularnewline
%\hline						   
% $T_{03}$  &    -               &   -0.15	   &   -0.15            &  -          &    -    \tabularnewline
\hline						   
 $U$       &    +10.06           &   +12.00	   &   +10.80           &  +3.04      &  +4.30 \tabularnewline
\hline						   
 $V_{01}$  &    +7.18            &   +8.00	   &   +7.02            &   -         &    -    \tabularnewline
\hline						   
 $V_{02}$  &    +5.11            &   +4.35	   &   +5.37            &   -         &    -    \tabularnewline
\hline						   
 $V_{03}$  &    +4.57            &   +4.57	   &   +4.57            &   -         &    -    \tabularnewline
\hline
\hline
\end{tabular}
\caption{Model Hamiltonian parameters (in eV) from various downfolding methods.}
\end{center}
\end{table*}
%In the first way, we consider a quantum state obtained from DMC
%with energy $\omega$ and ask what $U^{*}/t$ best describes 
%the double occupancy correlator at this spin state (remaining in 
%the same charge sector). In the second way, we assume that 
%the energy difference between this 
%state and the half filled ground state is described by a Hubbard model, whose 
%$U^{*}$ can be obtained completely from the knowledge of the density matrices

\subsection {Long range Hubbard model}
Having established the need for long-range interactions in benzene,
(based on discrepancies with respect to the ab-initio QMC correlators and energies),
we consider the case of the extended Hubbard (or Parisier-Pople-Parr) model.

In prior work, Bursill et al~\cite{Bursill} chose the form of the 
density-density interactions to be of the "Ohno form"
\begin{equation}
	V_{ij}= \frac{U}{\sqrt{1 + (\alpha r_{ij})^{2}}}
\end{equation}
where $\alpha$ is a fitting parameter and $r_{ij}$ is the spatial 
separation between nuclei. This parameterization 
has been widely used in the modelling of various organic polymers. 
More recently Schmalz~\cite{Schmalz} has discussed 
the accuracy of the Mataga-Nishimoto (MN) form, 
$V_{ij}= \frac{U}{1 + \alpha r_{ij}}$. 
Both forms insure that in the large $r_{ij}$ limit $V_{ij} \rightarrow \frac{1} {r_{ij}}$. 
The optimal parameters for these models are obtained by fitting 
their spectra to available experimental energies.
However, the results from both models differ: the Ohno-form is reported to be $U=10.06 $ eV while 
the MN form gives $U=8.32$ eV. 

We repeat similar analyses as that for the short-range model, in addition
to carefully looking at the variations in the 1- and 2-RDM matrix 
elements, shown in Fig.~\ref{fig:variations}. We find that $t_{02}$ 
is irrelevant in the charge-neutral sector and thus attempt to first fit 
to a model only with the nearest neighbor $t$ along with $U$, $V_{01}$ and $V_{02}$ 
($V_{03}$ is not needed as it simply sets the chemical ptential). 
We find that even in the $12$ eV window the maximum error 
between the DMC and fitted energies 
is now reduced to 0.15 eV shown in Fig.s~\ref{fig:expt}(b)

Figure~\ref{fig:expt}(c) shows a comparison of the
experimental energies with respect to various model 
energies. Both the Hubbard and the Ohno parameterizations 
can be considered to have small root mean square deviations: however, 
the fact that there are outliers with large deviations 
is indicative of their inadequacies. Our
long ranged fitted PPP Hamiltonian, accounts for these outliers
and can hence be considered to be accurate.

Our parameters and those of previous studies are shown in Table~\ref{tab:param}. 
While our results are most similar to the Ohno form, there are some 
crucial deviations (of the order of $0.3- 0.8$ eV) that 
improve the quality of our fitted energies. We emphasize that 
we have not provided any experimental inputs: rather we have used QMC data (energies and RDMs) 
from multiple states to directly to do the fitting. 
Further, we show that the fit is robust to the parameters, 
by estimating the errors from a Jacknife analysis in addition to 
the eigenvalues of the $A$ matrix used in our formalism.

\section{Graphene Solid}
We now consider graphene, a 2D solid of carbon atoms arranged on a honeycomb lattice. 
Graphene is a "wonder material" and has great potential technological applications, 
which explains the immense works devoted to understanding it thoroughly~\cite{graphene_review}.
That said, it is only recently that systematic studies to estimate the role of 
electron-electron interactions~\cite{Wehling_graphene,Schuler_graphene,Abbamonte} 
have been carried out. While its long-distance properties appear to be adequately 
described by a tight binding model, the short range features, 
crucial to study phenomena such as magnetism, require a more refined modeling.

\begin{figure}[htpb]
\centering
\includegraphics[width=1\linewidth]{./double_occupancy_graphene.pdf}
\caption{(Left) Double occupancy of the $\pi$ orbitals of $4\times4$ graphene for QMC, Non Interacting ($U^{*}/t = 0.0 $) 
and RPA ($U^{*}/t=1.6$). This data is best explained by a Hubbard model with $U^{*}/t=0.6$ 
(Right) Double occupancy vs $U^{*}/t$ for $L \times L$ graphene unit cells with periodic boundary conditions for $L=2,4,6,8$.
Very small finite size effects exist for $L \geq 4$.}
\label{fig:doubleocc} 
\end{figure}	

\begin{figure*}[htpb]
\centering
%\includegraphics[width=0.32\linewidth]{./rdm_graphene_hubbard.pdf}
\includegraphics[width=0.48\linewidth]{./nunu_graphene_hubbard.pdf}
\includegraphics[width=0.48\linewidth]{./nund_graphene_hubbard.pdf}
\caption{Comparison of ab-initio Quantum Monte Carlo 
and Hubbard model correlation functions (with respect to a reference site) 
for the half filled singlet ground state of graphene 
for various values of $U^{*}/t$. The optimal Hubbard model overestimates 
the attraction between electrons of opposite spins at short distances.}
\label{fig:corrs_graphene} 
\end{figure*}	

Early studies modelled graphene as a honeycomb lattice Hubbard model 
with a $U/t$ estimated to be $ \approx 3.8$. This would put graphene on the verge of a 
metal-insulator transition~\cite{Sorella_Tosatti, Sorella_Nature}. 
However, recent results by Wehling et al. realized the importance of 
long range interactions~\cite{Wehling_graphene},
which renormalize the on-site interaction to an effectively lower value. 
Schuler et al.~\cite{Schuler_graphene} reported this effective $U^{*}/t \sim 1.6 \pm 0.1$. 
This means graphene lies well in the semimetal phase of the Hubbard model. 

%The approach for downfolding from a lattice model with long range interactions (hamiltonian $H$) 
%to that with an onsite $U^{*}$ (hamiltonian $H^{*}$) 
%used the Peierls-Feynman-Bogoliubov variational principle, 
%which best matches the partition functions of the latter to the former. 
%This principle is closely related to matching the reduced density matrices of the ground state 
%in the zero temperature limit, albeit with subtle differences. 
%In particular, mapping one lattice model to another one is not what we refer to technically 
%refer as "downfolding" since there is no reduction of the number of electrons. 
%It must be emphasizes that our many body QMC calculations included all the $s$-like, 
%$\sigma$ and $\pi$ electrons of carbon. 
%The model Hamiltonian was downfolded onto the $\pi$-like orbital space.
Setting aside the question of uniquely determining all the long range interactions
in graphene, we ask what $U^{*}/t$ best describes our ground state Quantum Monte Carlo data. 
To do so, we first generated the $\pi$-like Wannier functions 
within QWalk~\cite{QWALK}, a representative of which has been shown in Fig.~\ref{fig:orbs}. 
Just as in the case of benzene, we used the fact 
that the effective strength of the Coulomb interaction 
$U^{*}/t$ is most sensitive to the 2-RDM element 
$\langle n_{i,\uparrow} n_{i,\downarrow}\rangle$. For the $4\times4$ 
unit cell with periodic boundary conditions, and using optimized Slater Jastrow wavefunctions, 
the VMC value is found to be $0.238$ and the mixed estimator (DMC) value is found to be 
$0.228$. This provided an extrapolated estimator value of $0.220 \pm 0.005$.

The $U^{*}/t$ is obtained from comparisons to 
lattice determinantal QMC calculations, which were carried out for 
sizes ranging from $2 \times 2$ to $8 \times 8$ to check for finite-size effects. 
As Fig.~\ref{fig:doubleocc} shows, the $2 \times 2$ unit cell is distinctly different from 
the larger unit cells, however, the finite-size errors in the 
double occupation correlator are negligible beyond sizes $L\geq 4$. 
The finite size effects for other short range correlation functions (not shown) are also negligible beyond 
$L \geq 4$.

Fig.~\ref{fig:corrs_graphene} shows the 1-RDM (for either spin at half filling) and the 
up density- down density correlator for various separations of orbitals. 
On the scale of Fig.~\ref{fig:corrs_graphene}(a) 
(and well within the accuracy of our calculations) the 1-RDM was captured 
well for all values of $U^{*}/t$ in the range from 0 to 2. However, as 
Fig.~\ref{fig:corrs_graphene}(b) shows the Hubbard model for large $U$ tends 
to exaggerate the the effective interaction 
between the two electron spin flavors at small separations. In particular, the 
nearest neighbor correlator is found to be in better agreement with $U^{*}/t \sim 0$ 
than any finite value. This, just like the case of benzene, suggests the need 
for longer range interactions in the model. 
There are also (extremely) small deviations between the \emph{ab-initio} QMC and the Hubbard model results 
at longer distances: the correlations do not depend significantly on $U^{*}/t$ and 
this data on its own does not rule out any model.
We also note that the rapid decay of the correlation functions (on the scale of only a 
few lattice spacings) is indicative of a large dielectric constant: however a systematic study 
is beyond the scope of this paper.
 
\section{Conclusion} 
We have demonstrated a method where \emph{ab-initio} Quantum Monte Carlo (QMC) 
data is used to fit simple effective Hamiltonians. We have elaborated 
on the fitting procedures and the intricacies of the QMC method 
needed to perform calculations. The limitations of the model 
were judged by assessing the quality of the fitted 
correlation functions (and more generally the 2-body density matrices) 
and energies. We expect this promising feature of our approach 
to be useful for constructing more refined models. 

For the benzene molecule, while 
the onsite Hubbard model with $U^{*}/t \approx 1.2$ 
was able to capture most features of the QMC ground state data, 
significant deviations were seen close to the bandwidth.
Allowing for longer range interactions, improved 
the quality of the correlation functions and the 
agreement of the vertical excited state energies 
with experiments. One could use this parameterization to calculate 
low frequency response functions and to test related 
semi-empirical methods.

Since QMC calculations use 
size-consistent wavefunctions for extended systems and 
scale favorably, we believe the type of calculations presented 
here will be a promising alternative to DFT-based downfolding approaches. 
Our demonstration for the single band model of graphene 
yielded an effective $U^{*}/t=1.1\pm0.1$ in the same 
range as a recently reported estimate based 
on the constrained-RPA method~\cite{Schuler_graphene}. In a future work,
we aim to use the excited states of graphene to accurately 
estimate its long range interactions and hence its dielectric constant. 
In more complicated materials, where the form of the Hamiltonian is 
possibly unclear, the form of the dominant terms can be obtained 
from canonical transformation theory followed with an 
accurate fit to QMC data. 

Finally, we remark that previously unsolved model Hamiltonians 
are now being accurately treated with tensor 
networks~\cite{Stoudenmire_White_2D_DMRG, Corboz_Rice_Troyer}. 
With parallel advances in the \emph{ab-initio} QMC simulation of high temperature 
superconductors~\cite{Wagner_cuprates, Krogel_cuprates}, a clear future direction 
is to deduce more refined models for these compounds, using the ideas discussed in the paper. 

\section{Acknowledgement} 
We thank  David Ceperley,  Cyrus Umrigar,  Garnet Chan,  Shiwei Zhang, 
 Steven White,  Bryan Clark, Norm Tubman and Victor Chua 
for discussions. We would also like to thank  Cyrus Umrigar for 
running very useful checks against his Quantum Monte Carlo code (CHAMP) in the early stages of 
this work. We acknowledge support from grant DOE FG02-12ER46875 (SciDAC). 

\bibliographystyle{prsty}
\bibliography{refs}

\end{document}
